\documentclass[a4paper]{article}
\usepackage{ctex}
\usepackage{amsmath}
\usepackage{amssymb}

\usepackage{geometry}  
\geometry{left=2.0cm, right=2.0cm, top=2.5cm, bottom=2.5cm}

\usepackage{braket}   

\title{谐振子的Unruh效应}
\author{吴梦之}

\begin{document}
    \maketitle

    \section{引言}
    \subsection{研究背景}
        Unruh效应是弯曲时空场论的一个有趣的结论。考虑闵氏时空中处于真空态的一个标量场,在任何惯性系中,它都处于真空态。然而,如果我们以匀加速观者的视角(Rindler时空)来考察它,我们将会发现这个场处于热平衡态,并且温度正比于加速度。Unruh效应有趣之处在于,我们所考虑的背景时空闵氏时空是平直的,这暗示我们另一种平直时空——Galileo时空可能也会有类似的效应,即匀加速观者观测Galileo时空中的物理现象,可能存在不平凡的物理效应。

        仿照格点QCD中的技巧,我们可以将时空格点化,这时我们所考虑的自由标量场就会退化为由无穷多个谐振子所构成的多体系统。最典型的由谐振子组成的多体系统是晶格振动系统(在正则量子化之后,可以称作声子系统),这启发我们可以思考晶格中是否也存在Unruh效应,即考虑零温声子场,匀加速观者是否能够观察到一个热能谱。

        为了研究无穷多谐振子的系统,我们可以从单体谐振子入手,即考虑一个一维谐振子在匀加速观者看到的效应。这本身也是一个很有趣的课题,近些年,一些课题组[2]做了一些有关量子力学中的参考系变换问题,得到了很多有趣的结论。文献[3]还讨论了匀加速观者对自由粒子的观测结果,并讨论了两种量子力学版本的等效原理。所以对于单个简谐振子,我们也可以考虑量子力学版本的等效原理,考察引力场中的谐振子和匀加速观者看到的无外场谐振子。

    \subsection{本文的主要结果}    

        \begin{itemize}
            \item 本文首先回顾了一般的幺正变换下Schrodinger方程的变换法则,并据此讨论了绝热规范势的物理意义,进而提出量子达朗贝尔原理,将绝热规范势诠释为惯性势。本文还利用Ehrenfest定理指出了惯性势的经典对应正是惯性力。
            \item 本文接下来针对谐振子,进行了进一步研究。本文首先指出参考系变换的Weyl平移算子在谐振子系统将给出相干态,并计算其Wigner函数,验证了它是一个以参考系速度运动的Gauss波包。本文之后探讨了参考系变换下的谐振子与Lewis-Riesenfeld理论下的运动谐振子,讨论“运动”参考系下的“静止”谐振子与“静止”参考系下的“运动”谐振子的等价性;这里“运动”和“静止”是指其平动自由度,而非振动自由度。
            \item 谐振子的Unruh效应
        \end{itemize}

    \section{量子力学中的参考系变换}

        关于量子力学中的参考系变换以及Galileo对称性,文献[1]做了很多讨论。不过由于[1]历史较为久远,在波动力学框架下研究问题,很多问题不能看得很透彻,甚至有一些错误。文献[2]对其则进行了更为深刻但较为简略的回顾。在此,本文将从他们工作的基础上入手。

    \subsection{Schrodinger方程的Galileo对称性、绝热规范势与量子达朗贝尔原理}
        考虑实验室参考系S中的一个态$\ket{\Psi}$,设S'参考系与S的坐标变换为$x'=x+\zeta(t), t'=t$。一般认为,S'对这个态的观测由一个幺正算符$\hat{U}_a$来刻画,量子态和Hamilton量分别变换为$\hat{U}_a\ket{\Psi}$和$\hat{U}_a\hat{H}\hat{U}_a^+$。然而,接下来的论述中我们将看到,在这样的变换下,Schrodinger方程的形式无法保持不变,为了抵消参考系变换带来的额外的效应,我们不得不修改Schrodinger方程或修改变换法则,这正是绝热规范势[5]和量子达朗贝尔原理[4]的物理意义。

        我们首先考虑不含时的参考系变换$x\rightarrow x'=x+\zeta$,其中$\zeta$为常数。这个参考系变换的意义就是将空间原点平移了一段距离$\zeta$,此时参考系变换算符$\hat{U}_a=\exp\big(\frac{i}{\hbar}\zeta\hat{p} \big)$,正是空间平移算子,而Hamilton量的变换法则是$\hat{H}\rightarrow \hat{H}'=\hat{U}_a\hat{H}\hat{U}_a^+$。于是在S'参考系中,$\hat{H}'\ket{\Psi'}=\hat{U}_a\hat{H}\hat{U}_a^+\hat{U}_a\ket{\Psi}=\hat{U}_a\hat{H}\ket{\Psi}=\hat{U}_a i\hbar\frac{\partial}{\partial t}\ket{\Psi}$,由于此处$\hat{U}_a$不含时,因此可以与$\frac{\partial}{\partial t}$交换次序,从而得到$\hat{H}'\ket{\Psi'}=i\hbar\frac{\partial}{\partial t}\hat{U}_a\ket{\Psi}=i\hbar\frac{\partial}{\partial t}\ket{\Psi'}$。也就是说,在这样的参考系变换下,Schrodinger方程的形式保持不变。从对称性的角度来看,这表明了Schrodinger方程具有空间平移对称性。我们可以进一步考虑Noether定理,这将会给出一个守恒量——动量。

        场论中,给定一个全局对称性,我们往往可以考虑相应的局域对称性(规范对称性),这里我们也可以考虑定域对称性。应当注意的是,定域对称性在场论层面理解为变换算符$\hat{U}_a$是时空坐标$x^\mu$的函数,而在量子力学层面则应理解为$\hat{U}_a$是时间t的函数,这某种程度上可以说是由于量子力学是0+1维场论。这时我们考察经过算符$\hat{U}_a$变换后的Schrodinger方程:$i\hbar\frac{\partial}{\partial t}\ket{\Psi'} = i\hbar\frac{\partial}{\partial t}\big(\hat{U}_a\ket{\Psi}\big) = i\hbar\frac{\partial\hat{U}_a}{\partial t}\ket{\Psi} + i\hbar\hat{U}_a\frac{\partial}{\partial t}\ket{\Psi} = i\hbar\frac{\partial\hat{U}_a}{\partial t}\ket{\Psi} + \hat{U}_a\hat{H}\ket{\Psi} = i\hbar\frac{\partial\hat{U}_a}{\partial t}\ket{\Psi} + \hat{H}'\ket{\Psi'}$。我们会发现Schrodinger方程经过$\hat{U}_a$变换后不再成立,存在多出来的一项$i\hbar\frac{\partial\hat{U}_a}{\partial t}\ket{\Psi} = i\hbar\frac{\partial\hat{U}_a}{\partial t}\hat{U}_a^+\ket{\Psi'}$。关于这一项,我们可以有两种思想来理解它。

        \begin{itemize}
        \item[1] 绝热规范势 
            
            我们可以参照规范场论来理解这一项。我们要求系统的运动方程具有$\hat{U}_a$所刻画的定域对称性(规范对称性),重申这里的"定域"是指时域上的定域,那么时域偏导数$\frac{\partial}{\partial t}$会破坏这种规范对称性。仿照规范场论,我们应当引入规范场(如电磁场、胶子场)来抵消额外的项,将偏导数改写为协变导数$\frac{D}{Dt}=\frac{\partial}{\partial t}+connection\ term$,并要求其与$\hat{U}_a$对易,即$\frac{D}{Dt}\big(\hat{U}_a\ket{\Psi}\big)=\hat{U}_a\frac{D}{Dt}\ket{\Psi}$。我们根据这一限制条件,不难得到协变导数的联络项应当为$-i\hbar\frac{\partial\hat{U}_a}{\partial t}\hat{U}_a^+$,这正是我们定域幺正变换在Schrodinger方程中带来的额外的一项。将普通导数改为协变导数之后,Schrodinger方程在S系和S'系中的形式都是

            \begin{equation}
                i\hbar\frac{D}{Dt}\ket{\Psi} = \hat{H}\ket{\Psi}
            \end{equation}

            其中$\frac{D}{Dt}=\frac{\partial}{\partial t}-i\hbar\frac{\partial\hat{U}_a}{\partial t}\hat{U}_a^+$。这个Schrodinger方程在$\hat{U}_a$变换下保持不变,即具有$\hat{U}_a$定域不变性。从这一角度来看,我们可以将这一项理解为规范对称性所带来的规范场,这也是为什么很多文献将其称为绝热规范势的原因。数学上来看,这一项代表了主纤维丛上的联络,称为Maurer-Cartan形式,它是一个几何上的量,因此将会贡献几何相位(Berry相位)。
        
        \item[2] 量子达朗贝尔原理
            
            我们可以重新剖析一下我们所遇到的问题。我们对系统的量子态和Hamilton量做了一个含时的幺正变换后,时域偏导数作用在这个幺正变换算符上会给出额外的一项,多出来的这一项破坏了运动方程——Schrodinger方程。类似的问题其实在经典物理层面也存在。在惯性系中,物体的经典运动方程是Newton第二定律$m\frac{d^2x}{dt^2}=F[x(t),\frac{dx}{dt}]$,其中$F[x(t),\frac{dx}{dt}]$是x(t)和$\frac{dx}{dt}$的泛函。但是在非惯性系中,物体的运动方程中存在一个惯性力,这个惯性力破坏了物体的运动方程——Newton第二定律。

            经典物理层面,我们采用达朗贝尔原理来处理惯性力,即重新定义$F[x(t),\frac{dx}{dt}]$,将惯性力吸收进去,这样我们就可以保持运动方程不变。量子力学层面我们也可以重新定义Hamilton量:
            
            \begin{equation}
                \hat{H}'=\hat{U}_a\hat{H}\hat{U}_a^+ + i\hbar\frac{\partial\hat{U}_a}{\partial t}\hat{U}_a^+
            \end{equation}

            这样在参考系变换下,Schrodinger方程的形式不发生改变。我们将重新定义Hamilton量来保证Schrodinger方程不变的这一想法称作量子达朗贝尔原理。当然,这一方法在传统文献中也常常使用,并且把重新定义的Hamilton量记作$\hat{H}_{eff}$
            
        \end{itemize}

        回顾以上讨论,我们所遇到的问题是当我们对系统做幺正变换时,系统可能会产生额外的效应,从而破坏运动方程——Schrodinger方程。我们希望系统具有这样的对称性,因此不希望运动方程遭到破坏,我们可以有两种方法来解决这个问题。第一种方法是改造运动方程,引入规范势,使新的运动方程在幺正变换下保持不变。第二种方法是我们将新的效应吸收到哈密顿量中,从而保持运动方程不变,称为量子达朗贝尔原理。

        应当指出的是,以上讨论原则上针对任意幺正变换,并不局限于参考系变换,可以是其它方式带来的幺正变换,例如态矢量的相位变换。由于参考系变换具有很强的特殊性,它所带来的绝热规范势具有经典对应,即惯性势(事实上,我们将会看到,绝热规范势中将有三个部分,其中一个部分对应于惯性势)。而考虑到经典的等效原理,惯性力局部地可以等效为引力,因此量子达朗贝尔原理为我们架起了连接规范理论和引力理论的一个桥梁,这将为我们理解引力的量子本性提供一个有趣的思路。正是由于刻画参考系变换的幺正变换的特殊性,我们将量子达朗贝尔原理区分为狭义和广义两种版本,其中狭义版本专指参考系变换的量子达朗贝尔原理,而广义版本则针对任意幺正变换。本文将主要针对狭义量子达朗贝尔原理。

    \subsection{$\hat{U}_a$的表达式与物理意义}

        根据以上讨论,我们理解了参考系变换下Schrodinger方程和Hamilton量的变换法则后,接下来我们考虑量子态的变换法则。量子态的变换由幺正算符$\hat{U}_a$来刻画,即若S参考系中观测到的态是$\ket{\Psi}$,则S’中观测到的态为$\hat{U}_a\ket{\Psi}$。根据文献[1,2],$\hat{U}_a$的一般表达式为:
    
        \begin{equation}
            \hat{U}_a=\exp\bigg(-\frac{i}{\hbar} m \frac{d \zeta}{dt} \hat{x}\bigg) \exp\bigg({\frac{i}{\hbar}\zeta(t) \hat{p}}\bigg) \exp\bigg(-\frac{i}{\hbar}\int_0^t\frac{1}{2}m(\frac{d \zeta}{d\tau})^2 d\tau\bigg)
        \end{equation}
        
        应注意,这三项中的前两项是算符,不能随意交换次序,第三项是数,因此可以随意摆放次序。为了更好地看清前两项的物理意义,我们分别考虑该算符作用在坐标算符和动量算符的本征态上,即考虑参考系变换对坐标本征态和动量本征态的影响。
        \begin{equation}
            \hat{U}_a \ket{x} = \exp\bigg(-\frac{i}{\hbar}m\frac{d \zeta}{dt}\ (x+\zeta)\bigg)\ \exp\bigg(-\frac{i}{\hbar}\int_0^t\frac{1}{2}m(\frac{d \zeta}{d\tau})^2 d\tau \bigg)\ \ket{x+\zeta}
        \end{equation}
        \begin{equation}
            \hat{U}_a \ket{p} = \exp\bigg(\frac{i}{\hbar}\zeta p \bigg)\ \exp\bigg(-\frac{i}{\hbar}\int_0^t\frac{1}{2}m(\frac{d \zeta}{d\tau})^2 d\tau \bigg)\ \ket{p+m\frac{d \zeta}{dt}}
        \end{equation}

        由以上两式可以知道,$\hat{U}_a$的作用是将坐标本征态平移一段距离$\zeta(t)$,将动量本征态增加动量$m\frac{d \zeta}{dt}$。这也符合我们对Galileo变换的经典物理图像——参考系变换之后,物理系统动能、动量和坐标零点会发生变化,而这三方面变化分别由以上三个相位来刻画。

        注意到当$[\hat{A}, [\hat{A}, \hat{B}]]=[\hat{B}, [\hat{A}, \hat{B}]]=0$时,$\hat{A}$和$\hat{B}$算符满足Baker-Hausdorff公式: $\exp(\hat{A}+\hat{B})=\exp\hat{A} \exp\hat{B} \exp(-\frac{[\hat{A}, \hat{B}]}{2})$, 因此我们可以将$\hat{U}_a$写成Weyl平移算子的形式:
        \begin{equation}
            \hat{U}_a=\exp\bigg(\frac{i}{\hbar}\bigg(\hat{p}\zeta(t)-m\frac{d\zeta}{dt}\hat{x} \bigg) \bigg)\ \exp\bigg(\frac{i}{\hbar}\frac{1}{2}m\zeta\frac{d\zeta}{dt} \bigg)\ \exp\bigg(-\frac{i}{\hbar}\int_0^t\frac{1}{2}m(\frac{d \zeta}{d\tau})^2 d\tau \bigg)
        \end{equation}
        第二项来自于$\hat{p}\zeta(t)$和$m\frac{d\zeta}{dt}\hat{x}$的对易关系,我们将其与第三项合并起来考虑,一并给出其物理意义。第三项的指数因子$-\frac{i}{\hbar}\int_0^t\frac{1}{2}m(\frac{d \zeta}{d\tau})^2 d\tau = -\frac{i}{\hbar}\int_{\zeta(0)\equiv 0}^{\zeta(t)}\frac{1}{2}m\frac{d \zeta}{d\tau} d\zeta = -\frac{i}{\hbar}\bigg(\frac{1}{2}m\zeta(t)\frac{d \zeta}{d\tau}-\int_0^t\frac{1}{2}m\zeta\frac{d^2\zeta}{d\tau^2} d\tau\bigg)$,不难看出分部积分后的第一项恰好可以与$\hat{U}_a$的第二项相消。于是$\hat{U}_a$可以进一步写成更紧凑的形式:

        \begin{equation}
            \hat{U}_a = \exp\bigg(\frac{i}{\hbar}\bigg(\hat{p}\zeta(t)-m\frac{d\zeta}{dt}\hat{x} \bigg) \bigg)\ \exp\bigg( -\frac{i}{\hbar}\int_0^t-\frac{1}{2}m\zeta\frac{d^2\zeta}{d\tau^2} d\tau\bigg)
        \end{equation}

        注意到$-m\frac{d^2\zeta}{dt^2}$代表了参考系变换时带来的经典惯性力,因此第二项的相位可以认为是惯性力做功所给出的。
        
        特别地,对于匀速和匀加速两种类型的参考系变换,幺正算符的形式分别为$\hat{U}_a=\exp\big(-\frac{i}{h}(\hat{p}vt-mv\hat{x})\big)$和$\hat{U}_a=\exp\big(\frac{i}{\hbar}\big(\frac{1}{2}at^2\hat{p}-mat\hat{x} \big)\big)\ \exp\big(\frac{1}{12}\frac{i}{\hbar}ma^2t^3\big)$。文献[3]利用路径积分方法研究了弱场近似下的引力场中的自由下落粒子,他们指出,引力的效应可以用一个Weyl平移算子以及一个额外的相位因子$\exp(i\frac{1}{12}mg^2t^3)\hat{W}(-mgt,-\frac{1}{2}gt^2)$描述,进而引力场中的自由下落粒子和自由粒子的概率幅分布是相同的,于是他们给出了量子力学等效原理的一个表述:引力场中自由下落的粒子和自由粒子的概率分布相同,与其质量无关。不难注意到,他们的Weyl平移算子和额外的相位严格地等于$\hat{U}_a$,因此他们的等效原理其实也可以加强为,不仅仅概率幅相同,其量子态也是完全相同的,与质量无关。

    \subsection{Ehrenfest定理以及狭义量子达朗贝尔原理}

        以上,我们讨论了Schrodinger方程、Hamilton量和量子态在参考系变换下的变换法则。为了更好地理解量子达朗贝尔原理,我们将通过Ehrenfest定理,来研究该原理的物理意义与经典对应。

        我们已经知道,一般参考系变换下的幺正算符为公式(2),我们接下来计算其惯性势。$\frac{\partial}{\partial t}\hat{U}_a = \big(-\frac{i}{\hbar}m\frac{d^2 \zeta}{dt^2} \hat{x}\big)\hat{U}_a + \hat{U_a}\big(\frac{i}{\hbar}\frac{d\zeta}{dt}\hat{p} - \frac{i}{\hbar}\frac{1}{2}m(\frac{d\zeta}{dt})^2\big)$。由于惯性势的第二个因子$\hat{U}_a^+$与$\hat{U}_a$相乘等于1,因此我们希望将$\hat{p}$与$\hat{U}_a$交换次序。根据对易关系$[\hat{x},\hat{p}]=i\hbar$,我们用归纳法不难知道$[\hat{x}^n,\hat{p}]=i\hbar n\hat{x}^{n-1}$,进而对于任意能够Taylor展开的算符函数$f(\hat{x},\hat{p})$, 都有$[\hat{f}, \hat{p}]=i\hbar\frac{\partial f}{\partial\hat{x}}$。所以$[\hat{U}_a, \hat{p}] = i\hbar\frac{\partial\hat{U}_a}{\partial\hat{x}} = i\hbar\big(-\frac{i}{\hbar}m\frac{d\zeta}{dt}\big)\hat{U}_a = m\frac{d\zeta}{dt}\hat{U}_a$。于是$\hat{U}_a\hat{p} = \hat{p}\hat{U}_a + [\hat{U}_a, \hat{p}] = \big(\hat{p}+m\frac{d\zeta}{dt}\big)\hat{U}_a$。这样,我们就得到了一般形式的惯性势如下:

        \begin{equation}
            \hat{V}_{iner} := i\hbar\frac{\partial\hat{U}_a}{\partial t}\hat{U}_a^+ = m \frac{d^2\zeta}{d t^2}\hat{x}-\frac{d\zeta}{dt}\hat{p}-\frac{1}{2}m\bigg(\frac{d\zeta}{dt}\bigg)^2
        \end{equation}

        量子力学中的Ehrenfest定理告诉我们$\frac{d}{dt}<\hat{p}>=-<\frac{\partial\hat{V}}{\partial \hat{x}}>$,即力学量的期望值服从Newton运动方程。那么经过参考系变换后,势能项增加了一个惯性势,于是Ehrenfest定理将会给出$-<\frac{\partial\hat{V}_{iner}}{\partial \hat{x}}>=-m \frac{d^2\zeta}{d t^2}$,这正是经典惯性力。这也说明,对于描述参考系变换的幺正算符,我们将它所带来的绝热规范势称为惯性势或者达朗贝尔势是合适的。

        绝热规范势$\hat{V}_{iner}$中的动量项$-\frac{d\zeta}{dt}\hat{p}$则应当与Hamilton量中的动能项$\frac{\hat{p}^2}{2m}$进行配平方,得到$\frac{\big(\hat{p}-m\frac{d\zeta}{dt}\big)^2}{2m}-\frac{1}{2}m\big(\frac{d\zeta}{dt}\big)$。第一项也是符合经典物理图像的,它代表了在S'系中观测物体运动,其动量会有$-m\frac{d\zeta}{dt}$的变化。第二项与$V_{iner}$中的$\frac{1}{2}m\big(\frac{d\zeta}{dt}\big)^2$合并,给出一个没有可观测效应的全局相位。因此,S'系中的Schrodinger方程的形式为:

        \begin{equation}
            i\hbar\frac{\partial}{\partial t}\ket{\Psi'} = \biggl(\frac{(\hat{p}-m\frac{d\zeta}{dt})^2}{2m}+\hat{V}(x+\zeta)-m\frac{d^2\zeta}{dt^2}\hat{x}-m(\frac{d\zeta}{dt})^2\biggr) \ket{\Psi'}
        \end{equation}
        
        特别地,我们可以考虑惯性系的boost变换$x\rightarrow x'=x+vt$,此时惯性势$\hat{V}_{iner}=v\hat{p}-\frac{1}{2}mv^2$,我们发现惯性势不含坐标算符$\hat{x}$,于是它所带来的惯性力等于0。这说明经典Newton运动方程在boost变换下具有不变性,从对称性的角度来看就是Newton方程满足Galileo对称性。此时S'参考系中的Hamilton量变为$\frac{\big(\hat{p}-m\frac{d\zeta}{dt}\big)^2}{2m}+V(\hat{x})-mv^2$,说明Schrodinger方程在boost变换下仅会多出势能零点的一个平移$-mv^2$,贡献一个没有可观测效应的全局相位,因此Schrodinger方程也具有Galileo对称性。

        综上,一般来讲,绝热规范势可以分成三个部分,其动量项应当与Hamilton量中的动能项配平方,代表了参考系变换下动量的平移。绝热规范势中正比于$\hat{x}$的部分代表了参考系变换导致的惯性力做功,因此是真正的惯性势,如果量子力学中等效原理成立,那么这一项将与引力产生联系。另外还剩一个势能零点的平移项,仅贡献一个没有可观测效应的全局相位。

        


    \section{参考系变换下的谐振子}

        以上我们对一般的参考系变换进行了讨论,接下来我们研究谐振子系统$\hat{H}=\frac{\hat{p}^2}{2m}+\frac{1}{2}m\omega^2\hat{x}^2$。谐振子系统中,$\hat{x}$和$\hat{p}$可以用产生湮灭算符表示$\hat{x}=\sqrt{\frac{\hbar}{2m\omega}}(\hat{a}^{+}+\hat{a})$,$\hat{p}=i\sqrt{\frac{m\hbar\omega}{2}}(\hat{a}^{+}-\hat{a})$。此时,Weyl平移算子$\hat{W}(a, b)=\exp(ia\hat{x}-ib\hat{p})$将会变成$\hat{D}(\alpha)=\exp(\alpha\hat{a}^{+}-\alpha^*\hat{a})$,其中$\alpha=b\sqrt{\frac{m\hbar\omega}{2}}+ia\sqrt{\frac{\hbar}{2m\omega}}$。$\hat{D}(\alpha)$算符称为谐振子的平移算符,其物理意义是将谐振子基态变换为本征值等于$\alpha$的相干态,即$\hat{a}\hat{D}(\alpha)\ket{0}=\alpha\hat{D}(\alpha)\ket{0}$。

        全体$\hat{D}(\alpha)$模掉一个全局相位后构成一个群$\{\hat{D}(\alpha):\mathcal{H}\to\mathcal{H}|\alpha\in\mathbb{C}\}/\sim$,其中等价关系$\sim$定义为$\hat{D}(\alpha)\sim\hat{D}(\beta) \Leftrightarrow \hat{D}(\alpha)=\hat{D}(\beta)\exp(i\phi),\ \phi\in\mathbb{R}$。这个群满足$\hat{D}(\alpha+\beta)=\hat{D}(\alpha)\hat{D}(\beta)$,其中“=”指模掉全局相位意义下的相等,进而不难验证它与$(\mathbb{C},+)$同构,因此可以说$\hat{D}(\alpha)$事实上是$(\mathbb{C},+)$群在谐振子Hilbert空间上的群表示,进而相干态可以认为是复平面上的点作用在基态上,从而相干态与复平面上的点是一一对应的。

        由于$\hat{W}(a,b)$与$\hat{D}(\alpha)$具有一一对应关系,因此在模掉一个全局相位的意义下,$\hat{W}(a,b)$事实上代表了$(\mathbb{R}^2,+)$群在Hilbert空间上的群表示,并且相干态上与$\mathbb{R}^2$的点一一对应。(a, b)的物理意义是经典相空间,其中a代表相空间动量,b代表相空间坐标。因此$\hat{W}(a,b)$代表了经典相空间作为一个加法群作用在量子态空间上,相空间的一个点对应于一个平移算符,相空间中随时间演化的一条径迹则对应于随时间演化的一族平移算符。将这族平移算子作用在谐振子基态上,就对应于基态的演化。参考系变换对应的Weyl平移算子为$\hat{W}(-m\frac{d\zeta}{dt}, -\zeta)$,接下来我们将计算坐标空间和动量空间波函数以及谐振子的Wigner函数,最后将看到$\hat{W}(-m\frac{d\zeta}{dt}, -\zeta)$意味着基态的中心位置将以$-\zeta(t)$进行平移运动,而动量中心值将以$-\frac{d\zeta}{dt}$平移。这也是符合经典物理图像的。

    \subsection{平动参考系中谐振子的波函数与Wigner函数}

        由以上分析论证,我们知道参考系变换将谐振子基态变成一个相干态,其中$\alpha=-\zeta\sqrt{\frac{m\omega}{2\hbar}}-im\frac{d\zeta}{dt}\sqrt{\frac{1}{2m\hbar\omega}}$。波函数则等于:

        \begin{equation}
            \begin{split}   
                \Psi'_0(x,t) &= \bigg(\frac{m\omega}{\pi\hbar}\bigg)^{1/4} \exp\biggl(i\Im(\alpha)\sqrt{\frac{2m\omega}{\hbar}}x\biggr) \exp(-i\Im(\alpha)\Re(\alpha)) \exp\biggl(-\frac{m\omega}{2\hbar} \bigl(x-\Re(\alpha)\sqrt{\frac{2\hbar}{m\omega}}\bigr)^2 \biggr) \exp(-\frac{i}{\hbar}E_0 t) \\
                    &= \bigg(\frac{m\omega}{\pi\hbar}\bigg)^{1/4} \exp\biggl(-\frac{i}{\hbar}m\frac{d\zeta}{dt}x\biggr) \exp(-\frac{i}{2\hbar}m\frac{d\zeta}{dt}\cdot\zeta) \exp\biggl(-\frac{m\omega}{2\hbar} \bigl(x+\zeta\bigr)^2 \biggr) \exp(-\frac{i}{\hbar}E_0 t)
            \end{split}
        \end{equation}

        其中$\Re(\alpha)$和$\Im(\alpha)$分别代表$\alpha$的实部和虚部。根据波函数的表达式,我们来计算S'系观察者看到的谐振子基态的Wigner函数

        \begin{equation}
            \begin{split}
                W(x, p) &= \int_{-\infty}^\infty \frac{d\xi}{2\pi}\bra{x+\frac{1}{2}\xi}\hat{\rho}\ket{x-\frac{1}{2}\xi} e^{-\frac{i}{\hbar}p\xi} \\
                    &= \int_{-\infty}^\infty\frac{d\xi}{2\pi} \Psi(x+\frac{1}{2}\xi)\Psi^*(x-\frac{1}{2}\xi) e^{-\frac{i}{\hbar}p\xi}
            \end{split}
        \end{equation}

        这里,$\hat{\rho}=\ket{\alpha}\bra{\alpha}$是S'系观察者看到的谐振子基态的密度矩阵。经过一定量的计算后,可以得到

        \begin{equation}\label{Wigner}
            \begin{split}
                W_{S'}(x, p) &= \sqrt{\frac{m\omega}{\pi\hbar}} \sqrt{\frac{\hbar}{2\pi m\omega}} \exp\biggl(-\frac{m\omega}{\hbar}\bigl(x-\Re(\alpha)\sqrt{\frac{2\hbar}{m\omega}}\bigr)^2\biggr) \exp\biggl(-\frac{1}{2m\hbar\omega}\bigl(p-\Im(\alpha)\sqrt{2m\hbar\omega}\bigr)^2\biggr) \\
                    &= \frac{1}{\sqrt{2}\pi} \exp\biggl(-\frac{m\omega}{\hbar}\bigl(x+\zeta\bigr)^2\biggr) \exp\biggl(-\frac{1}{2m\hbar\omega}\bigl(p+m\frac{d\zeta}{dt}\bigr)^2\biggr)
            \end{split}
        \end{equation}

        我们知道,对Wigner函数的一个变量求积分可以得到另一个变量的概率分布,即Wigner函数的边缘分布性质:$\bra{x}\hat{\rho}\ket{x}=\int dp W(x, p)$以及$\bra{p}\hat{\rho}\ket{p}=\int dx W(x, p)$。我们对(\ref{Wigner})积分,即可得到S'参考系中观察者看到的坐标和动量的概率分布:

        \begin{equation}
            |\Psi'_0(x,t)|^2 = \sqrt{\frac{m\omega}{\pi\hbar}} \exp\bigg(\frac{m\omega}{\hbar}(x+\zeta)^2\bigg)
        \end{equation}

        \begin{equation}
            |\Psi'_0(p,t)|^2 = \sqrt{\frac{m\hbar\omega}{\pi}} \exp\biggl(-\frac{1}{2m\hbar\omega}\bigl(p+m\frac{d\zeta}{dt}\bigr)^2\biggr)
        \end{equation}
        
        说明在S'系观察者看来,谐振子基态是一个运动的Gauss波包,其中心位置$-\zeta(t)$正是平动参考系的原点的位置。





    \subsection{实验室参考系下的平动谐振子}
        我们可以考虑实验室系中外力作用下的一个平动谐振子$\hat{H}=\frac{\hat{p}^2}{2m}+\frac{1}{2}m\omega^2\big(\hat{x}-\zeta(t)\big)^2$,这样的谐振子在量子操控领域很有用,代表了我们输运一个谐振子态。经典力学层面下,我们预期静止参考系下的平动谐振子,应该等同于平动参考系下无平动自由度的谐振子。接下来我们将研究量子情形下这个结论是否仍然成立。

        根据Lewis-Riesenfeld理论[6, 7],这个Hamilton量具有精确解:

        \begin{equation}
            \begin{split}  
                \Psi(x,t)& =\frac{1}{\sqrt{2^n n!}}\bigg(\frac{m\omega}{\pi\hbar}\bigg)^{1/4} \exp\bigg(-\frac{i}{\hbar}\int_0^t \bigg(E_n+\frac{1}{2}m\big(\frac{dx_c}{d\tau}\big)^2\bigg)d\tau\bigg) \\
                    & \times\exp\bigg(-\frac{m\omega}{2\hbar}(x-x_c)^2\bigg) \exp\bigg(\frac{i}{\hbar}m\frac{dx_c}{dt}x\bigg) H_n\bigg(\sqrt{\frac{m\omega}{\hbar}}(x-x_c)\bigg)
            \end{split}
        \end{equation}
    
        其中,$x_c$由方程$\frac{d^2}{dt^2}x_c+\omega^2(x_c-\zeta)=0$确定,

    \subsection{匀加速参考系变换下的谐振子}

    \section{谐振子的Unruh效应}


    \section*{参考文献}
    \begin{itemize}
        \item[1] F. Giacomini, et al. Nature Communications (2019)10:494
        \item[2] D. M. Greenberg. Am. J. Phys. 47.1, 35-38 (1979)
        \item[3] C. Anastopoulos, et al. arxiv: 1707.04526
        \item[4] 杨冠卓. 绝热捷径中的若干量子热力学问题研究. 上海:上海大学, 2019
        \item[5] M. Kolodrubetz, et al. arxiv: 1602.01062
        \item[6] H. R. Lewis. Jr. Math. Phys. 9, 1976(1968) 
        \item[7] H. R. Lewis, W. B. Riesenfeld. Jr. Math. Phys. 10, 1458(1969)
        \item[8] 
    \end{itemize}

\end{document}