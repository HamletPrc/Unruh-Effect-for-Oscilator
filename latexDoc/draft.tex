\documentclass[a4paper]{article}
\usepackage{ctex}

\usepackage{geometry}
\geometry{left=2.0cm, right=2.0cm, top=2.5cm, bottom=2.5cm}

\usepackage{braket}

\title{匀加速参考系变换下的谐振子}
\author{吴梦之}

\begin{document}
    \maketitle
    \section{引言}
        Unruh效应是弯曲时空场论的一个有趣的结论。考虑闵氏时空中处于真空态的一个标量场,在任何惯性系中,它都处于真空态。然而,如果我们以匀加速观者的视角(Rindler时空)来考察它,我们将会发现这个场处于热平衡态,并且温度正比于加速度。Unruh效应有趣之处在于,我们所考虑的背景时空闵氏时空是平直的,这暗示我们另一种平直时空——伽利略时空可能也会有类似的效应,即匀加速观者观测伽利略时空中的物理现象,可能存在不平凡的物理效应。

        仿照色动力学中的技巧,我们可以将时空格点化,这时我们所考虑的自由标量场就会退化为由无穷多个谐振子所构成的多体系统。最典型的由谐振子组成的多体系统是晶格振动系统(在正则量子化之后,可以称作声子系统),这启发我们可以思考晶格中是否也存在Unruh效应,即考虑零温声子场,匀加速观者是否能够观察到一个热能谱。

        为了研究无穷多谐振子的系统,我们可以从单体谐振子入手,即考虑一个一维谐振子在匀加速观者看到的效应。这本身也是一个很有趣的课题,近些年,一些课题组[2]做了一些有关量子力学中的参考系变换问题,得到了很多有趣的结论。文献[3]还讨论了匀加速观者对自由粒子的观测结果,并讨论了两种量子力学版本的等效原理。所以对于单个简谐振子,我们也可以考虑量子力学版本的等效原理,考察引力场中的谐振子和匀加速观者看到的无外场谐振子。

        因此本工作将主要研究匀加速观者观测到的单个谐振子,主要研究以下几个问题:
        \begin{itemize}
            \item 验证谐振子的伽利略对称性,即对于匀速直线运动的观察者来说,是否有额外的效应
            \item 推导匀加速观者看到的谐振子的哈密顿量,并求解相应薛定谔方程(或者海森堡方程或者传播子),比较静止系下的谐振子,并理解经典层面的惯性力的量子力学对应
            \item 讨论一维谐振子是否存在Unruh效应,并理解为什么有或者为什么没有
            \item 推导引力场中的谐振子波函数,与匀加速观者观察到的谐振子作比较,讨论量子力学版本的等效原理
        \end{itemize}

        至于晶格振动的Unruh效应、等效原理等方面的工作则放到后续的工作中。


    \section{参考系变换的一般性分析}
        关于量子力学中的参考系变换以及伽利略对称性,文献[1]做了很多讨论。不过由于[1]历史较为久远,在薛定谔绘景下研究问题,很多问题不能看得很透彻,甚至有一些错误。文献[2]对其则进行了更为深刻但较为简略的回顾。在此,本文将从他们的工作入手。

        \subsection{态的变换}
        
        考虑实验室参考系S中的一个态$\ket{\Psi}$,设S’参考系与S的坐标变换为$x'=x+\zeta(t), t'=t$。则S’对这个态的观测的观测由一个幺正算符$\hat{U}_a$来刻画,即S’中观测到的态为$\hat{U}_a\ket{\Psi}$。根据文献[2],这个幺正算符的一般表达式为:
        \begin{equation}
            \hat{U}_a=\exp\bigg(-\frac{i}{\hbar} m \frac{d \zeta}{dt} \hat{x}\bigg) \exp\bigg({\frac{i}{\hbar}\zeta(t) \hat{p}}\bigg) \exp\bigg(-\frac{i}{\hbar}\int_0^t\frac{1}{2}m(\frac{d \zeta}{d\tau})^2 d\tau\bigg)
        \end{equation}
        
        应注意,这三项中的前两项是算符,不能随意交换次序,第三项是数,因此可以随意摆放次序。$\hat{U}_a$ 的第三项的物理意义是,参考系变换所带来的额外动能$E_k=\frac{1}{2}m(\frac{d \zeta}{d\tau})^2$所给出的态随时间演化的积分相位。为了更好地看清前两项的物理意义,我们分别考虑该算符作用在坐标算符和动量算符的本征态上,即考虑参考系变换对坐标本征态和动量本征态的影响。
        \begin{equation}
            \hat{U}_a \ket{x} = \exp\bigg(-\frac{i}{\hbar}m\frac{d \zeta}{dt}\ (x+\zeta)\bigg)\ \exp\bigg(-\frac{i}{\hbar}\int_0^t\frac{1}{2}m(\frac{d \zeta}{d\tau})^2 d\tau \bigg)\ \ket{x+\zeta}
        \end{equation}
        \begin{equation}
            \hat{U}_a \ket{p} = \exp\bigg(\frac{i}{\hbar}\zeta p \bigg)\ \exp\bigg(-\frac{i}{\hbar}\int_0^t\frac{1}{2}m(\frac{d \zeta}{d\tau})^2 d\tau \bigg)\ \ket{p+m\frac{d \zeta}{dt}}
        \end{equation}

        这里应当注意,由于$\hat{x}$和$\hat{p}$不对易,因此$\hat{U}_a$的两部分要保持次序。
        由以上两式可以知道,$\hat{U}_a$的作用是将坐标本征态平移一段距离$\zeta(t)$,将动量本征态增加动量$m\frac{d \zeta}{dt}$。这也符合我们对伽利略变换的经典物理图像,参考系变换之后,物理系统动能、动量和坐标零点会发生变化,而这三方面变化分别由以上三个相位来刻画。

        注意到当$[\hat{A}, [\hat{A}, \hat{B}]]=[\hat{B}, [\hat{A}, \hat{B}]]=0$时,$\hat{A}$和$\hat{B}$算符满足Baker-Hausdorff公式:$\exp(\hat{A}+\hat{B})=\exp\hat{A} \exp\hat{B} \exp(-\frac{[\hat{A}, \hat{B}]}{2})$, 因此我们可以将$\hat{U}_a$写成更紧凑的形式:
        \begin{equation}
            \hat{U}_a=\exp\bigg(\frac{i}{\hbar}\bigg(\hat{p}\zeta(t)-m\frac{d\zeta}{dt}\hat{x} \bigg) \bigg)\ \exp\bigg(\frac{i}{\hbar}\frac{1}{2}m\zeta\frac{d\zeta}{dt} \bigg)\ \exp\bigg(-\frac{i}{\hbar}\int_0^t\frac{1}{2}m(\frac{d \zeta}{d\tau})^2 d\tau \bigg)
        \end{equation}
        此时,第一项是一个Weyl平移算子,第二项来自于$\hat{p}\zeta(t)$和$m\frac{d\zeta}{dt}\hat{x}$的对易关系,它所贡献的相位可以理解为位力定理给出的相位。
        
        特别地,对于匀速和匀加速两种类型的参考系变换,幺正算符的形式分别为$\hat{U}_a=\exp\big(-\frac{i}{h}(\hat{p}vt-mv\hat{x})\big)$和$\hat{U}_a=\exp\big(\frac{i}{\hbar}\big(\frac{1}{2}at^2\hat{p}-mat\hat{x} \big)\big)\ \exp\big(\frac{1}{12}\frac{i}{\hbar}ma^2t^3\big)$。文献[3]计算了引力弱场近似下自由粒子的传播子,与无引力情形相比,相差了一个Weyl平移算子以及一个额外的相位,总的效应恰好等于这里的$\hat{U}_a$。这一定程度反映了等效原理的量子版本,即引力场局部等效于加速参考系所带来的惯性力。

        \subsection{哈密顿量和薛定谔方程的变换与量子达朗贝尔原理}

            即便在经典力学中,哈密顿量在参考系变换下一般不是不变量。考虑牛顿第二定律$F=ma$,仅在惯性系变换下牛顿第二定律仍然成立。而在非惯性系中,物体会受到


        \subsection{力学量的变换与Ehrenfest定理}



    \section{惯性系中的谐振子}
        特别地,我们考虑惯性系变换。此时,其中 代表两个参考系之间的相对速度,于是 将会具有形式:
        
        注意到当算符 满足对易关系 时,有公式
        
        我们可以得到 的简化形式:
        
        可以看到,这时 是一个Weyl变换的算符。但意义是什么,暂时不知道。另外,文献[3]在引力系统中,对于自由粒子得到了类似的结果,说明了引力与匀加速观测者的局部等价性,即量子等效原理。
        更特别地,考虑本文所关心的谐振子系统,可以把坐标算符和动量算符用产生湮灭算符来表达:,于是 算符将会具有形式:
        
        其中 。这个算符正是谐振子系统的平移算符,这意味着,在S’看来,S系中的谐振子基态将会变成相干态。考虑到相干态的物理图像是一个运动的高斯波包,于是S’所观测到的S系基态谐振子就是一个运动的高斯波包,这也是符合我们的经典图像和预期的。
    


    \section*{参考文献}
    \begin{itemize}
        \item[1] F. Giacomini, et al. Nature Communications (2019)10:494
        \item[2] D. M. Greenberg. Am. J. Phys. 47.1, 35-38 (1979)
        \item[3] C. Anastopoulos, et al. arxiv: 1707.04526
        \item[4] 杨冠卓. 绝热捷径中的若干量子热力学问题研究. 上海:上海大学, 2019
    \end{itemize}
\end{document}