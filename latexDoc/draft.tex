\documentclass{article}
\usepackage{ctex}
\begin{document}
    \section{引言}

    \section{参考系变换的一般性分析}
    关于量子力学中的参考系变换以及伽利略对称性,文献[1]做了很多讨论。不过由于[1]历史较为久远,在薛定谔绘景下研究问题,很多问题不能看得很透彻,甚至有一些错误。文献[2]对其则进行了更为深刻但较为简略的回顾。在此,本文将从他们的工作入手。
    
    考虑实验室参考系S中的一个态$|\Psi>$,设S’参考系与S的坐标变换为$x'=x+\zeta(t), t'=t$。则S’对 的观测由一个幺正算符来刻画,即S’中观测到的态为 。根据文献[2],这个幺正算符的一般表达式为:
    
    应注意,这三项中的前两项是算符,不能随意交换次序,第三项是数,因此可以随意摆放次序。这个幺正算符的第三项是参考系变换带来的额外动能所给出的态随时间的演化相位。为了更好地看清前两项的物理意义,我们分别考虑该算符作用在坐标算符和动量算符的本征态上,即考虑参考系变换对坐标本征态和动量本征态的影响。
    
    
    这里应当注意,由于 和 不对易,因此 的两部分要保持次序。
    由以上两式可以知道, 的作用是将坐标本征态平移一段距离,将动量本征态增加一定的动量。这也符合我们对伽利略变换的经典物理图像,参考系变换之后,物理系统动能、动量和坐标零点会发生变化,而这三方面变化分别由以上三个相位来刻画。
    
    特别地,我们考虑惯性系变换。此时,其中 代表两个参考系之间的相对速度,于是 将会具有形式:
    
    注意到当算符 满足对易关系 时,有公式
    
    我们可以得到 的简化形式:
    
    可以看到,这时 是一个Weyl变换的算符。但意义是什么,暂时不知道。另外,文献[3]在引力系统中,对于自由粒子得到了类似的结果,说明了引力与匀加速观测者的局部等价性,即量子等效原理。
    更特别地,考虑本文所关心的谐振子系统,可以把坐标算符和动量算符用产生湮灭算符来表达:,于是 算符将会具有形式:
    
    其中 。这个算符正是谐振子系统的平移算符,这意味着,在S’看来,S系中的谐振子基态将会变成相干态。考虑到相干态的物理图像是一个运动的高斯波包,于是S’所观测到的S系基态谐振子就是一个运动的高斯波包,这也是符合我们的经典图像和预期的。
    
    \section{惯性系中的谐振子}

\end{document}