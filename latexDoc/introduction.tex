\documentclass{article}
\usepackage{ctex}

\usepackage{geometry}  
\geometry{left=2.0cm, right=2.0cm, top=2.5cm, bottom=2.5cm}

\begin{document}
    \section{引言}

    \subsection{研究背景}
        参考系变换在物理学中具有基石性的地位,我们一般要求物理状态以及物理规律不依赖于参考系或者观察者。在经典力学中,Galileo原理要求力学规律在惯性系中具有相同的形式。然而结合经典电磁学理论时,绝对时空观与Galileo原理发生了冲突,Einstein因此抛弃了绝对时空观,并基于新的协变性原理建立了狭义相对论[1],使力学和电动力学规律在参考系变换下具有协变性。Galileo原理和狭义相对性原理都是针对惯性系的,而对于一般参考系,Einstein提出等效原理[2],把非惯性系带来的惯性力局部地等效为引力,从而构建广义协变性原理和广义相对论[3]。在现代场论中,我们对参考系变换的要求也是严苛的——Poincare对称性必须被满足。用于刻画运动规律的作用量和拉氏量必须是Lorentz标量,各种场和单粒子态的分类实际上就是对Lorentz群的不可约表示进行分类[4,5]。

        量子力学中的参考系变换是近年来一个有趣的课题。由于观察者可以是经典观察者,也可以是量子观察者,因此参考系变换就分为经典参考系变换和量子参考系变换[7]。由于Unruh效应考虑的是经典观测者,因此本文将主要考虑经典参考系变换。一般的参考系变换可以分解平动和转动两个部分。平动变换的刻画是简单的,只需要知道原点的运动,即可知道所有点的运动,因此我们可以用一个坐标原点处的幺正算符来刻画一个平动变换。转动变换则比较难于刻画,首先是因为转动参考系变换每个点的运动都不同,因此坐标空间每个点都需要一个算符来刻画该点的参考系变换下的运动;其次同一个三维转动变换可以有多种表达式。正是由于转动问题的复杂性,本文将主要关注Galileo时空下平动参考系下的问题。

        Unruh效应[6]是广义相对论和弯曲时空场论中一个有趣的结论。考虑闵氏时空中处于真空态的一个标量场,在任何惯性系中,它都处于真空态。然而,如果我们以匀加速观者的视角(Rindler时空)来考察它,我们将会发现这个场处于热平衡态,并且温度正比于加速度。Unruh效应有趣之处在于,我们所考虑的背景时空闵氏时空是平直的,这暗示我们另一种平直时空——Galileo时空可能也会有类似的效应,即匀加速观者观测Galileo时空中的物理现象,可能存在不平凡的物理效应。场的物理图像可以认为是无穷多的小谐振子所构成的系统,这启发我们思考单个谐振子是否也存在Unruh效应,即匀加速观者是否能够观测到一个热能谱。

    \subsection{本文主要工作}

        本文将首先回顾经典平动参考系变换的一般理论,对一般的平动参考系变换乃至一般的幺正变换构建量子达朗贝尔原理,从Yang-Mills理论的角度分析其物理意义,并借助Ehrenfest定理阐明量子达朗贝尔原理的经典对应,顺便简单讨论惯性势和量子力学版本的等效原理。

        本文接下来将基于平动参考系变换的一般理论,研究参考系变换下的谐振子。本文指出了用于刻画参考系变换的Weyl平移算子实际上是经典相空间作为一个加法群作用在量子态空间上,并据此证明了谐振子基态在参考系变换下是一个相干态,坐标空间中是一个逆着参考系原点运动的Gauss波包。本文还对比了“运动”参考系下的“静止”谐振子和“静止”参考系下的“运动”谐振子,指出了“运动”谐振子的绝对性。
\end{document}