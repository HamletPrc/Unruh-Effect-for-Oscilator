\documentclass[a4paper]{article}
\usepackage{ctex}
\usepackage{amsmath}
\usepackage{amssymb}

\usepackage{geometry}  
\geometry{left=2.0cm, right=2.0cm, top=2.5cm, bottom=2.5cm}

\usepackage{braket}

\usepackage{amsthm}
\newtheorem*{conclusion}{结论}
\newtheorem*{D'Alembert}{量子达朗贝尔原理}

\title{平动参考系变换下的谐振子}
\author{吴梦之}

\begin{document}
    \maketitle

    \begin{center}
        \fontsize{18pt}{0}摘\quad\quad 要
    \end{center}

        本文研究了量子力学中的经典平动参考系变换及其在谐振子系统中的应用。本文首先论证了平动参考系变换下量子态的变换由Weyl平移算子刻画,它是经典相空间作为一个加法群在Hilbert空间上的群表示。之后本文研究了参考系变换下Hamilton量的变换法则,指出了此时的绝热规范势代表了惯性力,本文然后简单讨论了量子版本的达朗贝尔原理和等效原理。本文接下来根据量子态和Hamilton量的变换法则,研究了平动参考系变换下的谐振子,着重讨论了其基态的变换和Hamilton量的变换,论证了匀速参考系变换下基态仍为基态,而非匀速参考系变换下由于惯性力的作用,基态将变为相干态。本文最后简单讨论了此结论在量子调控领域的潜在应用,以及Galileo时空下是否存在Unruh效应。

    \textbf{关键词} \ \ \ 经典平动参考系变换 \ \ \ Weyl平移算子 \ \ \ 量子达朗贝尔原理 \ \ \ 谐振子相干态



    \section{引言}

    \subsection{研究背景}

        Unruh效应[?]是广义相对论和弯曲时空场论中一个有趣的结论。考虑闵氏时空中处于真空态的一个标量场,在任何惯性系中,它都处于真空态。然而,如果我们以匀加速观者的视角(Rindler时空)来考察它,我们将会发现这个场处于热平衡态,并且温度正比于加速度。Unruh效应有趣之处在于,我们所考虑的背景时空闵氏时空是平直的,这暗示我们另一种平直时空——Galileo时空可能也会有类似的效应,即匀加速观者观测Galileo时空中的物理现象,可能存在不平凡的物理效应。场的物理图像可以认为是无穷多的小谐振子所构成的系统,这启发我们思考单个谐振子是否也存在Unruh效应,即匀加速观者是否能够观测到一个热平衡态。

        于是我们就需要考虑量子力学中的参考系变换,近年来这也是一个有趣的前沿课题。观察者可以是经典观察者,也可以是量子观察者,因此参考系变换就分为经典参考系变换和量子参考系变换[???]。由于Unruh效应考虑的是经典观测者,因此本文将主要考虑经典参考系变换。

    \subsection{本文的结构安排与主要成果}

        本文第二章首先回顾了平动参考系变换下量子态的一般性变换法则,讨论幺正变换矩阵每一项的物理意义。然后借助Weyl平移算子和Weyl-Heisenberg群,本文证明了在群表示论的观点下,平动参考系变换是经典相空间作为一个加法群作用于量子态空间上,即为经典相空间的群表示。

        本文第三章则考虑参考系变换对Schrodinger方程和Hamilton量的影响,回顾绝热规范势的物理意义,指出参考系变换算子所诱导的绝热规范势的经典物理意义是惯性力,由此构建了量子力学版本的达朗贝尔原理,并简单讨论量子力学中的等效原理。

        本文第四章则基于以上两章的结论,考察平动参考系变换下的谐振子,特别是谐振子基态。本文将Weyl平移算子表示为产生湮灭算符的形式,证明了参考系变换后的基态将变成一个中心位置随参考系运动的Gauss波包,并通过计算这个Gauss波包的Wigner函数及其空间分布和动量分布,来进一步验证这个结论。本文接下来在变换后的Hamilton表象下研究这个Gauss波包,证明了匀速参考系变换情形下,运动的Gauss波包是新Hamilton量的瞬时基态;而在非匀速情形下,由于惯性力的作用,Gauss波包变成了新Hamilton量的相干态。



    \section{平动参考系中的量子态}
        
        我们假设有两个观察者S和S'观测同一个量子态,不妨假定S是静止观察者,S'是运动观察者。一般来讲,S和S'之间参考系变换可以分解平动和转动两个部分。平动变换的刻画是简单的,我们只需要知道运动参考系S'原点的运动情况,即可知道所有点的参考系变换关系,因此对于平动参考系变换,我们只需要刻画原点的运动即可完全确定参考系变换法则。而转动变换则比较难于刻画,困难至少包括两个个方面:
        
        \begin{itemize}
            \item 转动参考系变换中存在一个不动点,那么在这个不动点附近,转动变换某种程度上变成了一种内部自由度。例如原点处的一个自旋,在转动参考系变换下,只有自旋发生改变,而自旋是这个系统的内部自由度。
            \item 用Euler角来刻画转动时,同一个三维转动变换可以有多种表达式。这意味着我们要对全体Euler转动进行分类,每一个分类中只取一个表达式作为真实的转动变换。而这种分类所导致的等价关系,很可能会诱导出新的规范场。
        \end{itemize}
        
        正是由于转动问题的复杂性,本文将主要关注平动参考系变换。这一节我们分析量子态的变换,下一节我们将分析Schrodinger方程的变换,并引出绝热规范势。
        
    \subsection{参考系变换的幺正算符}
    
        在非相对论时空,即S'和S的时间t是一致的。设S'的原点O'相对于S的原点O的运动为$\xi(t)$,即S'中原点在S中的坐标为$\xi(t)$。则一个经典物体P在S'中的坐标为$x'=x-\xi(t)$,动量为$p'=p-m\frac{d\xi}{dt}$。那么在量子力学中,我们应当要求其期望值满足此性质,即
    
        \begin{equation}
            \begin{split}
                <x'>&=<x-\xi(t)>      \\
                <p'>&=<p-m\frac{d\xi}{dt}>
            \end{split}
        \end{equation}
    
        设S观测到的态为$\ket{\Psi}$,S'中观测到的态为$\ket{\Psi'}$。我们认为S和S'中Hilbert空间整体是不发生变化的,变化的仅仅是态矢量,因此存在一个算符$\hat{U}$联系$\ket{\Psi}$和$\ket{\Psi'}$,即$\ket{\Psi'}=\hat{U}\ket{\Psi}$。简单起见,我们考虑一维情形。由于坐标算符和动量算符的全体本征态分别构成态空间的一组基,因此我们只需要考虑坐标算符和动量算符的本征态在参考系变换下的变换关系,即可知道任意态在参考系变换下的变换关系,即得到$\hat{U}$的表达式。根据经典参考系变换的结果,我们可以要求

        \begin{equation}
            \hat{U}\ket{x}=e^{i\phi(x,t)}\ket{x-\xi(t)}
        \end{equation}

        其中,含时待定相位$\phi$允许坐标算符的e指数$e^{ik\hat{x}}$给出一个贡献。由于$\ket{x-\xi}=e^{\frac{i}{\hbar}\hat{p}\xi(t)}\ket{x}$,因此参考系变换算符$\hat{U}$应当包含一个空间平移算子$e^{\frac{i}{\hbar}\hat{p}\xi(t)}$。而对于待定相位$e^{ik\hat{x}}$部分,我们可以通过考虑$\hat{U}$作用在动量本征态$\ket{p}$上来推导。

        \begin{equation}
            \hat{U}\ket{p}=e^{i\phi(p,t)}\ket{p-m\frac{d\xi}{dt}}
        \end{equation}

        此时,含时待定相位$\phi$有一部分来自于空间平移算子$e^{\frac{i}{\hbar}\hat{p}\xi(t)}$。由于$\ket{p-m\frac{d\xi}{dt}}=e^{-\frac{i}{\hbar}m\frac{d\xi}{dt}\hat{x}}\ket{p}$,因此$\hat{U}$还应当包含一个动量平移算子$e^{-\frac{i}{\hbar}m\frac{d\xi}{dt}\hat{x}}$。因此参考系变换算符的表达式应该为

        \begin{equation}\label{Utmp}
            \hat{U}(t)=e^{-\frac{i}{\hbar}m\frac{d\xi}{dt}\hat{x}}e^{i\phi(t)} e^{\frac{i}{\hbar}\hat{p}\xi(t)}
        \end{equation}

        应当注意,$e^{\frac{i}{\hbar}\hat{p}\xi(t)}$和$e^{-\frac{i}{\hbar}m\frac{d\xi}{dt}\hat{x}}$的次序是重要的,交换次序时会相差一个Weyl相位。在(\ref{Utmp})中,我们允许一个额外的相位$\phi(t)$,根据文献[?],这个相位为$i\phi(t)=-\frac{i}{\hbar}\int_0^t\frac{1}{2}m(\frac{d \xi}{d\tau})^2 d\tau$。因此用于描述参考系变换的算符的完整表达式是

        \begin{equation}\label{U}
            \hat{U}(t) = \exp\biggl[-\frac{i}{\hbar}m\frac{d\xi}{dt}\hat{x}\biggr] \exp\biggl[\frac{i}{\hbar}\hat{p}\xi(t)\biggr] \exp\biggl[-\frac{i}{\hbar}\int_0^t\frac{1}{2}m(\frac{d \xi}{d\tau})^2 d\tau\biggr]
        \end{equation}

        (\ref{U})的物理意义是明确的,$\hat{U}$的作用是将坐标本征态平移一段距离$-\xi(t)$,将动量本征态增加动量$-m\frac{d \xi}{dt}$,这是符合参考系变换的经典物理图像的。

    \subsection{Weyl平移算子与经典相空间}

        注意到当$[\hat{A}, [\hat{A}, \hat{B}]]=[\hat{B}, [\hat{A}, \hat{B}]]=0$时,$\hat{A}$和$\hat{B}$算符满足Baker-Hausdorff公式: $\exp(\hat{A}+\hat{B})=\exp\hat{A} \exp\hat{B} \exp(-\frac{[\hat{A}, \hat{B}]}{2})$, 因此我们可以将(\ref{U})改写成Weyl平移算子的形式:

        \begin{equation}
            \hat{U}=\exp\bigg[\frac{i}{\hbar}\bigg(\hat{p}\xi(t)-m\frac{d\xi}{dt}\hat{x} \bigg) \bigg]\ \exp\bigg(\frac{i}{\hbar}\frac{1}{2}m\xi\frac{d\xi}{dt} \bigg)\ \exp\bigg(-\frac{i}{\hbar}\int_0^t\frac{1}{2}m(\frac{d \xi}{d\tau})^2 d\tau \bigg)
        \end{equation}

        其中,$\hat{W}(a,b):=\exp[\frac{i}{\hbar}(a\hat{x}-b\hat{p})]$称为Weyl平移算子。第二项来自于$\hat{p}\xi(t)$和$m\frac{d\xi}{dt}\hat{x}$的对易关系,我们将其与第三项合并起来考虑,一并给出其物理意义。第三项的指数因子$-\frac{i}{\hbar}\int_0^t\frac{1}{2}m(\frac{d \xi}{d\tau})^2 d\tau = -\frac{i}{\hbar}\int_{\xi(0)\equiv 0}^{\xi(t)}\frac{1}{2}m\frac{d \xi}{d\tau} d\xi = -\frac{i}{\hbar}\bigg(\frac{1}{2}m\xi(t)\frac{d \xi}{d\tau}-\int_0^t\frac{1}{2}m\xi\frac{d^2\xi}{d\tau^2} d\tau\bigg)$,不难看出分部积分后的第一项恰好可以与$\hat{U}$的第二项相消。于是$\hat{U}$可以进一步写成更紧凑的形式:

        \begin{equation}\label{U-Weyl}
            \hat{U} = \exp\bigg[\frac{i}{\hbar}\bigg(\hat{p}\xi(t)-m\frac{d\xi}{dt}\hat{x} \bigg) \bigg]\ \exp\bigg( -\frac{i}{\hbar}\int_0^t-\frac{1}{2}m\xi\frac{d^2\xi}{d\tau^2} d\tau\bigg)
        \end{equation}

        (\ref{U-Weyl})可以简写为$\hat{U}=\hat{W}(-m\frac{d\xi}{dt},-\xi)e^{i\phi}$。注意到$-m\frac{d^2\xi}{dt^2}$代表了参考系变换时带来的经典惯性力,因此(\ref{U-Weyl})第二项的相位可以认为是惯性力做功所给出的。特别地,对于匀速和匀加速两种类型的参考系变换,幺正算符的形式分别为$\hat{U}=\exp\big(-\frac{i}{h}(\hat{p}vt-mv\hat{x})\big)$和$\hat{U}=\exp\big(\frac{i}{\hbar}\big(\frac{1}{2}at^2\hat{p}-mat\hat{x} \big)\big)\ \exp\big(\frac{1}{12}\frac{i}{\hbar}ma^2t^3\big)$。文献[?]利用路径积分方法研究了弱场近似下的引力场中的自由下落粒子,他们指出,引力的效应可以用一个Weyl平移算子以及一个额外的相位因子$\exp(i\frac{1}{12}mg^2t^3)\hat{W}(-mgt,-\frac{1}{2}gt^2)$描述,进而引力场中的自由下落粒子和自由粒子的概率幅分布是相同的,于是他们给出了量子力学等效原理的一个表述:引力场中自由下落的粒子和自由粒子的概率分布相同,与其质量无关。不难注意到,他们的Weyl平移算子和额外的相位严格地等于$\hat{U}$,因此他们的等效原理其实也可以加强为,不仅仅概率幅相同,其量子态也是完全相同的,与质量无关。

        应当指出,$\hat{W}(a_1, b_1)\hat{W}(a_2, b_2)=\hat{W}(a_1+a_2, b_1+b_2)e^{i\phi}$,其中相角$\phi=\frac{a_1b_2-a_2b_1}{\hbar}$来自于Baker-Hausdorff公式。因此全体Weyl平移算符在模掉一个全局相位的意义下构成一个群$\mathcal{W}$,即$\hat{W}(a_1, b_1)\hat{W}(a_2, b_2)\sim\hat{W}(a_1+a_2, b_1+b_2)$。由于参数a, b取值为全体实数$\mathbb{R}$,因此这个群事实上同构于$(\mathbb{R}^2,+)$群,因此我们可以认为Weyl平移算符实际上是$(\mathbb{R}^2,+)$群对于量子态空间的群作用。物理上来看,$\hat{W}(-m\frac{d\xi}{dt}, -\xi)$的两个参数$-m\frac{d\xi}{dt}$和$-\xi(t)$代表了参考系变换所导致的经典相空间的平移。所以我们就得到了本节最重要的结论:

        \begin{conclusion}
            量子力学中平动参考系变换的作用是,经典相空间作为一个加法群对量子态空间的群作用,其中相空间的群运算是参考系变换所导致的坐标和动量的平移。
        \end{conclusion}
    
        最后我们指出,这个群实际上是Weyl-Heisenberg群相对于U(1)群的商群。Weyl-Heisenberg群对应的Lie代数可以给出广义相干态[???],在第四章我们将会看到,对于谐振子系统,Weyl平移算子可以把其基态变成一个相干态。

    \section{运动方程的Galileo对称性与绝热规范势}

        以上我们讨论了参考系变换下量子态的变换法则$\ket{\Psi}\to\ket{\Psi'}=\hat{U}\ket{\Psi}$,我们指出了变换算符$\hat{U}$的形式是一个Weyl平移算子乘一个全局相位(\ref{U-Weyl}),并且可以认为是作为加法群的经典相空间在量子态空间上的群表示。接下来我们来研究Schrodinger方程在参考系变换下的变化。

        Schrodinger方程是量子力学中的运动方程,我们应当要求它在参考系变换下具有某种意义上的不变性。正如我们要求经典力学的Newton第二定律Galileo对称性、量子场论的各种场和单粒子态必须满足Poincare或Lorentz对称性一样,运动方程必须要满足时空对称性,否则我们就会遇到1905年之前的电动力学的困境——只有在以太参考系下Maxwell方程组才成立,其它参考系中运动方程必须要大幅修正才能描述物理现象。然而接下来我们将看到,由于参考系变换$\hat{U}(t)$含时,Schrodinger方程中的$\partial_t$作用其上将会带来额外效应,这正是绝热规范势[?]的物理意义。

        \subsection{绝热规范势的引入}

        简单起见,我们首先考虑不含时的参考系变换$x\rightarrow x'=x-\xi$,其中$\xi$为常数。这个参考系变换的意义就是将空间原点平移了一段距离$\xi$,此时参考系变换算符$\hat{U}=\exp\big(\frac{i}{\hbar}\xi\hat{p} \big)$,正是空间平移算子,而Hamilton量的变换法则是$\hat{H}\rightarrow \hat{H}'=\hat{U}\hat{H}\hat{U}^\dagger$。于是在S'参考系中,$\hat{H}'\ket{\Psi'}=\hat{U}\hat{H}\hat{U}^\dagger\hat{U}\ket{\Psi}=\hat{U}\hat{H}\ket{\Psi}=\hat{U} i\hbar\frac{\partial}{\partial t}\ket{\Psi}$,由于此处$\hat{U}$不含时,因此可以与$\frac{\partial}{\partial t}$交换次序,从而得到$\hat{H}'\ket{\Psi'}=i\hbar\frac{\partial}{\partial t}\hat{U}\ket{\Psi}=i\hbar\frac{\partial}{\partial t}\ket{\Psi'}$。也就是说,在这样的参考系变换下,Schrodinger方程的形式保持不变。从对称性的角度来看,这表明了Schrodinger方程具有空间平移对称性。我们可以进一步考虑Noether定理,这将会给出一个守恒量——动量。

        场论中,给定一个全局对称性,我们往往可以考虑相应的局域对称性(规范对称性),这里我们也可以考虑定域对称性。应当注意的是,定域对称性在场论层面理解为变换算符$\hat{U}$是时空坐标$x^\mu$的函数,而在量子力学层面则应理解为$\hat{U}$是时间t的函数,这某种程度上可以说是由于量子力学是0+1维场论。这时我们考察经过算符$\hat{U}$变换后的Schrodinger方程:$i\hbar\frac{\partial}{\partial t}\ket{\Psi'} = i\hbar\frac{\partial}{\partial t}\big(\hat{U}\ket{\Psi}\big) = i\hbar\frac{\partial\hat{U}}{\partial t}\ket{\Psi} + i\hbar\hat{U}\frac{\partial}{\partial t}\ket{\Psi} = i\hbar\frac{\partial\hat{U}}{\partial t}\ket{\Psi} + \hat{U}\hat{H}\ket{\Psi} = i\hbar\frac{\partial\hat{U}}{\partial t}\ket{\Psi} + \hat{H}'\ket{\Psi'} = i\hbar\frac{\partial\hat{U}}{\partial t}\hat{U}^\dagger\ket{\Psi'} + \hat{H}'\ket{\Psi'}$。我们会发现Schrodinger方程经过$\hat{U}$变换后不再成立,存在多出来的一项
        
        \begin{equation}
            i\hbar\frac{\partial\hat{U}}{\partial t}\hat{U}^\dagger\sim i\hbar\partial_t(\log\hat{U})\ket{\Psi'}
        \end{equation}

        关于多出来的这一项,我们可以从两个方面来理解它。

        \begin{itemize}
        \item[1] 绝热规范势 
            
            我们可以参照规范场论来理解这一项。我们要求系统的运动方程具有$\hat{U}$所刻画的定域对称性(规范对称性),重申这里的"定域"是指时域上的定域,那么时域偏导数$\frac{\partial}{\partial t}$会破坏这种规范对称性。仿照规范场论,我们应当引入规范场(如电磁场、胶子场)来抵消额外的项,将偏导数改写为协变导数$\frac{D}{Dt}=\frac{\partial}{\partial t}+connection\ term$,并要求其与$\hat{U}$对易,即$\frac{D}{Dt}\big(\hat{U}\ket{\Psi}\big)=\hat{U}\frac{D}{Dt}\ket{\Psi}$。我们根据这一限制条件,不难得到协变导数的联络项应当为$-i\hbar\frac{\partial\hat{U}}{\partial t}\hat{U}^\dagger$,这正是我们定域幺正变换在Schrodinger方程中带来的额外的一项。将普通导数改为协变导数之后,Schrodinger方程在S系和S'系中的形式都是

            \begin{equation}
                i\hbar\frac{D}{Dt}\ket{\Psi} = \hat{H}\ket{\Psi}
            \end{equation}

            其中$\frac{D}{Dt}=\frac{\partial}{\partial t}-i\hbar\frac{\partial\hat{U}}{\partial t}\hat{U}^\dagger$。这个Schrodinger方程在$\hat{U}$变换下保持不变,即具有$\hat{U}$定域不变性。从这一角度来看,我们可以将这一项理解为规范对称性所带来的规范场,这也是为什么很多文献将其称为绝热规范势的原因。数学上来看,这一项代表了主纤维丛上的联络,称为Maurer-Cartan形式,它是一个几何上的量,因此将会贡献几何相位(Berry相位)。
        
        \item[2] 量子达朗贝尔原理
            
            我们可以重新剖析一下我们所遇到的问题。我们对系统的量子态和Hamilton量做了一个含时的幺正变换后,时域偏导数作用在这个幺正变换算符上会给出额外的一项,多出来的这一项破坏了运动方程——Schrodinger方程。类似的问题其实在经典物理层面也存在。在惯性系中,物体的经典运动方程是Newton第二定律$m\frac{d^2x}{dt^2}=F[x(t),\frac{dx}{dt}]$,其中$F[x(t),\frac{dx}{dt}]$是x(t)和$\frac{dx}{dt}$的泛函。但是在非惯性系中,物体的运动方程中存在一个惯性力,这个惯性力破坏了物体的运动方程——Newton第二定律。

            经典物理层面,我们采用达朗贝尔原理来处理惯性力,即重新定义$F[x(t),\frac{dx}{dt}]$,将惯性力吸收进去,这样我们就可以保持运动方程不变。量子力学层面我们也可以重新定义Hamilton量:
            
            \begin{equation}
                \hat{H}'=\hat{U}\hat{H}\hat{U}^\dagger + i\hbar\frac{\partial\hat{U}}{\partial t}\hat{U}^\dagger
            \end{equation}

            这样在参考系变换下,Schrodinger方程的形式不发生改变。我们将重新定义Hamilton量来保证Schrodinger方程不变的这一想法称作

            \begin{D'Alembert}
                对系统做参考系变换后,Schrodinger方程会因此多出一个规范势,将规范势吸收到Hamilton量中以保持Schrodinger方程形式不变
            \end{D'Alembert}
            
        \end{itemize}

        回顾以上讨论,我们所遇到的问题是当我们对系统做含时幺正变换时,系统会产生额外的效应,从而破坏运动方程——Schrodinger方程。我们希望系统具有这样的对称性,因此不希望运动方程遭到破坏,我们可以有两种方法来解决这个问题。第一种方法是改造运动方程,引入规范势,使新的运动方程在幺正变换下保持不变。第二种方法是我们将新的效应吸收到Hamilton量中,从而保持运动方程不变,称为量子达朗贝尔原理。

        应当指出的是,以上讨论原则上针对任意幺正变换,并不局限于参考系变换,可以是其它方式带来的幺正变换,例如态矢量的相位变换。由于参考系变换具有很强的特殊性,它所带来的绝热规范势具有经典对应,即惯性势(事实上,我们将会看到,绝热规范势中将有三个部分,其中一个部分对应于惯性势)。而考虑到经典的等效原理,惯性力局部地可以等效为引力,因此量子达朗贝尔原理为我们架起了连接规范理论和引力理论的一个桥梁,这将为我们理解引力的量子本性提供一个有趣的思路。正是由于刻画参考系变换的幺正变换的特殊性,我们将量子达朗贝尔原理区分为狭义和广义两种版本,其中狭义版本专指参考系变换的量子达朗贝尔原理,而广义版本则针对任意幺正变换。

    \subsection{量子达朗贝尔原理的经典物理意义}

        以上,我们讨论了Schrodinger方程、Hamilton量和量子态在参考系变换下的变换法则。为了更好地理解量子达朗贝尔原理,我们将通过Ehrenfest定理,来研究该原理的物理意义与经典对应。

        我们已经知道,一般参考系变换下的幺正算符为(\ref{U})或(\ref{U-Weyl}),我们接下来计算其惯性势。$\frac{\partial}{\partial t}\hat{U} = \big(-\frac{i}{\hbar}m\frac{d^2 \xi}{dt^2} \hat{x}\big)\hat{U} + \hat{U}\big(\frac{i}{\hbar}\frac{d\xi}{dt}\hat{p} + \frac{i}{\hbar}\frac{1}{2}m(\frac{d\xi}{dt})^2\big)$。由于惯性势的第二个因子$\hat{U}^\dagger$与$\hat{U}$相乘等于1,因此我们希望将$\hat{p}$与$\hat{U}$交换次序。根据对易关系$[\hat{x},\hat{p}]=i\hbar$,我们用归纳法不难知道$[\hat{x}^n,\hat{p}]=i\hbar n\hat{x}^{n-1}$,进而对于任意能够Taylor展开的算符函数$f(\hat{x},\hat{p})$, 都有$[\hat{f}, \hat{p}]=i\hbar\frac{\partial f}{\partial\hat{x}}$。所以$[\hat{U}, \hat{p}] = i\hbar\frac{\partial\hat{U}}{\partial\hat{x}} = i\hbar\big(-\frac{i}{\hbar}m\frac{d\xi}{dt}\big)\hat{U} = m\frac{d\xi}{dt}\hat{U}$。于是$\hat{U}\hat{p} = \hat{p}\hat{U} + [\hat{U}, \hat{p}] = \big(\hat{p}+m\frac{d\xi}{dt}\big)\hat{U}$。这样,我们就得到了一般形式的惯性势如下:

        \begin{equation}
            \hat{V}_{iner} := i\hbar\frac{\partial\hat{U}}{\partial t}\hat{U}^\dagger = m \frac{d^2\xi}{d t^2}\hat{x}-\frac{d\xi}{dt}\hat{p}+\frac{1}{2}m\bigg(\frac{d\xi}{dt}\bigg)^2
        \end{equation}

        量子力学中的Ehrenfest定理告诉我们$\frac{d}{dt}<\hat{p}>=-<\frac{\partial\hat{V}}{\partial \hat{x}}>$,即力学量的期望值服从Newton运动方程。那么经过参考系变换后,势能项增加了一个惯性势,于是Ehrenfest定理将会给出$-<\frac{\partial\hat{V}_{iner}}{\partial \hat{x}}>=-m \frac{d^2\xi}{d t^2}$,这正是经典惯性力。这也说明,对于描述参考系变换的幺正算符,我们将它所带来的绝热规范势称为惯性势或者达朗贝尔势是合适的。

        绝热规范势$\hat{V}_{iner}$中的动量项$-\frac{d\xi}{dt}\hat{p}$则应当与Hamilton量中的动能项$\frac{\hat{p}^2}{2m}$进行配平方,得到$\frac{\big(\hat{p}-m\frac{d\xi}{dt}\big)^2}{2m}-\frac{1}{2}m\big(\frac{d\xi}{dt}\big)$。第一项也是符合经典物理图像的,它代表了在S'系中观测物体运动,其动量会有$-m\frac{d\xi}{dt}$的变化。第二项与$V_{iner}$中的$\frac{1}{2}m\big(\frac{d\xi}{dt}\big)^2$恰好相消。因此,S'系中的Schrodinger方程的形式为:

        \begin{equation}
            i\hbar\frac{\partial}{\partial t}\ket{\Psi'} = \biggl(\frac{(\hat{p}-m\frac{d\xi}{dt})^2}{2m}+\hat{V}(x-\xi)-m\frac{d^2\xi}{dt^2}\hat{x}\biggr) \ket{\Psi'}
        \end{equation}
        
        特别地,我们可以考虑惯性系的boost变换$x\rightarrow x'=x+vt$,此时惯性势$\hat{V}_{iner}=v\hat{p}-\frac{1}{2}mv^2$,我们发现惯性势不含坐标算符$\hat{x}$,于是它所带来的惯性力等于0。这说明经典Newton运动方程在boost变换下具有不变性,从对称性的角度来看就是Newton方程满足Galileo对称性。此时S'参考系中的Hamilton量变为$\frac{\big(\hat{p}-m\frac{d\xi}{dt}\big)^2}{2m}$,说明Schrodinger方程也具有Galileo对称性。

        综上,一般来讲,绝热规范势可以分成两个部分,其动量项应当与Hamilton量中的动能项配平方,代表了参考系变换下动量的平移。绝热规范势中正比于$\hat{x}$的部分代表了参考系变换导致的惯性力做功,因此是真正的惯性势,如果量子力学中等效原理成立,那么这一项将与引力产生联系。


    
    \section{平动参考系变换下的谐振子}

        以上我们对一般的参考系变换进行了讨论,接下来我们研究谐振子系统$\hat{H}=\frac{\hat{p}^2}{2m}+\frac{1}{2}m\omega^2\hat{x}^2$。谐振子系统中,$\hat{x}$和$\hat{p}$可以用产生湮灭算符表示$\hat{x}=\sqrt{\frac{\hbar}{2m\omega}}(\hat{a}^\dagger+\hat{a})$,$\hat{p}=i\sqrt{\frac{m\hbar\omega}{2}}(\hat{a}^\dagger-\hat{a})$。此时,Weyl平移算子$\hat{W}(a,b)=\exp[\frac{i}{\hbar}(a\hat{x}-b\hat{p})]$将会变成$\hat{D}(\alpha)=\exp(\alpha\hat{a}^\dagger-\alpha^*\hat{a})$的形式,其中$\alpha=b\sqrt{\frac{m\omega}{2\hbar}}+i\frac{a}{\sqrt{2m\hbar\omega}}$。$\hat{D}(\alpha)$算符称为谐振子的平移算符,其物理意义是将谐振子基态变换为本征值等于$\alpha$的相干态,即$\hat{a}\hat{D}(\alpha)\ket{0}=\alpha\hat{D}(\alpha)\ket{0}$。

        我们已经知道全体Weyl平移算子在模掉一个全局相位的意义下构成一个群$\mathcal{W}\simeq(\mathbb{R}^2,+)$,相应地,全体$\hat{D}(\alpha)$模掉一个全局相位后构成一个群$\{\hat{D}(\alpha):\mathcal{H}\to\mathcal{H}|\alpha\in\mathbb{C}\}/\sim$,其中$\mathcal{H}$是量子态空间,等价关系$\sim$定义为$\hat{D}(\alpha)\sim\hat{D}(\beta) \Leftrightarrow \hat{D}(\alpha)=\hat{D}(\beta)\exp(i\phi),\ \phi\in\mathbb{R}$。这个群满足$\hat{D}(\alpha+\beta)=\hat{D}(\alpha)\hat{D}(\beta)$,其中“=”指模掉全局相位意义下的相等,进而它与$(\mathbb{C},+)$同构,因此可以说$\hat{D}(\alpha)$事实上是$(\mathbb{C},+)$群在谐振子Hilbert空间上的群表示,进而相干态可以认为是复平面上的点作用在基态上,从而相干态与复平面上的点是一一对应的。由于基态变换的特殊性,我们接下来将着重研究S系中的基态在S'系中的变化。

    \subsection{参考系变换下的谐振子基态}

        不过这个相干态的演化与普通的相干态的演化是不同的。对于一个普通的相干态$\hat{\alpha}$,它随时间演化的规律是$\ket{\alpha;t}=e^{-i\frac{1}{2}\omega t}\ket{\alpha e^{-i\omega t}}$。但是这里,$\ket{0';t}$的演化来自于两个方面,一个是基态$\ket{0}$自身随时间的演化,另一个是S'参考系随时间有变化,因此参考系变换算符$\hat{U}$随时间有变化。综合两个方面的影响,S'系中观察到的S系谐振子基态为

        \begin{equation}
            \ket{0';t} = e^{-i\frac{1}{2}\omega t}\ket{\alpha_t}, \ \alpha_t=-\xi\sqrt{\frac{m\omega}{2\hbar}}-im\frac{d\xi}{dt}\sqrt{\frac{1}{2m\hbar\omega}}
        \end{equation}

        波函数则等于:

        \begin{equation}
            \begin{split}   
                \Psi'_0(x,t) &= \bigg(\frac{m\omega}{\pi\hbar}\bigg)^{1/4} \exp\biggl[i\Im(\alpha_t)\sqrt{\frac{2m\omega}{\hbar}}x\biggr] \exp\biggl[-i\Im(\alpha_t)\Re(\alpha_t)\biggr] \exp\biggl[-\frac{m\omega}{2\hbar} \bigl(x-\Re(\alpha_t)\sqrt{\frac{2\hbar}{m\omega}}\bigr)^2 \biggr] \exp(-i\frac{1}{2}\omega t) \\
                    &= \bigg(\frac{m\omega}{\pi\hbar}\bigg)^{1/4} \exp\biggl[-\frac{m\omega}{2\hbar} \bigl(x+\xi\bigr)^2-\frac{i}{\hbar}m\frac{d\xi}{dt}x\biggr] \exp\biggl(-\frac{i}{2\hbar}m\frac{d\xi}{dt}\cdot\xi-i\frac{1}{2}\omega t \biggr)
            \end{split}
        \end{equation}

        其中$\Re(\alpha_t)=-\xi\sqrt{\frac{m\omega}{2\hbar}}$和$\Im(\alpha_t)=-m\frac{d\xi}{dt}\sqrt{\frac{1}{2m\hbar\omega}}$分别代表$\alpha_t$的实部和虚部。根据波函数的表达式,我们来计算S'系观察者看到的谐振子基态的Wigner函数

        \begin{equation}
            \begin{split}
                W(x, p) &= \int_{-\infty}^\infty \frac{d\xi}{2\pi}\bra{x+\frac{1}{2}\xi}\hat{\rho}\ket{x-\frac{1}{2}\xi} e^{-\frac{i}{\hbar}p\xi} \\
                    &= \int_{-\infty}^\infty\frac{d\xi}{2\pi} \Psi(x+\frac{1}{2}\xi)\Psi^*(x-\frac{1}{2}\xi) e^{-\frac{i}{\hbar}p\xi}
            \end{split}
        \end{equation}

        这里,$\hat{\rho}=\ket{\alpha}\bra{\alpha}$是S'系观察者看到的谐振子基态的密度矩阵。经过一定量的计算后,可以得到

        \begin{equation}\label{Wigner}
            \begin{split}
                W_{S'}(x, p) &= \sqrt{\frac{m\omega}{\pi\hbar}} \sqrt{\frac{\hbar}{2\pi m\omega}} \exp\biggl(-\frac{m\omega}{\hbar}\bigl(x-\Re(\alpha)\sqrt{\frac{2\hbar}{m\omega}}\bigr)^2\biggr) \exp\biggl(-\frac{1}{2m\hbar\omega}\bigl(p-\Im(\alpha)\sqrt{2m\hbar\omega}\bigr)^2\biggr) \\
                    &= \frac{1}{\sqrt{2}\pi} \exp\biggl(-\frac{m\omega}{\hbar}\bigl(x+\xi\bigr)^2\biggr) \exp\biggl(-\frac{1}{2m\hbar\omega}\bigl(p+m\frac{d\xi}{dt}\bigr)^2\biggr)
            \end{split}
        \end{equation}

        我们知道,对Wigner函数的一个变量求积分可以得到另一个变量的概率分布,即Wigner函数的边缘分布性质:$\bra{x}\hat{\rho}\ket{x}=\int dp W(x, p)$以及$\bra{p}\hat{\rho}\ket{p}=\int dx W(x, p)$。我们对(\ref{Wigner})积分,即可得到S'参考系中观察者看到的坐标和动量的概率分布:

        \begin{equation}
            |\Psi'_0(x,t)|^2 = \sqrt{\frac{m\omega}{\pi\hbar}} \exp\bigg(\frac{m\omega}{\hbar}(x+\xi)^2\bigg)
        \end{equation}

        \begin{equation}
            |\Psi'_0(p,t)|^2 = \sqrt{\frac{m\hbar\omega}{\pi}} \exp\biggl(-\frac{1}{2m\hbar\omega}\bigl(p+m\frac{d\xi}{dt}\bigr)^2\biggr)
        \end{equation}
        
        说明在S'系观察者看来,谐振子基态是一个运动的Gauss波包,其中心位置$-\xi(t)$正是平动参考系的原点的位置,动量的中心值$-m\frac{d\xi}{dt}$也正是平动参考系带来的额外的动量。



    \subsection{S'观者的Hamilton表象}

        以上我们已经说明了在S'系中看来,S系中谐振子的基态会变成一个运动的高斯波包,相对于S系的Hamilton量$\hat{H}=\frac{\hat{p}^2}{2m}+\frac{1}{2}m\omega^2\hat{x}^2$而言,这是一个相干态。然而,对于S'系的观察者,这个高斯波包应当在$\hat{H}'=\hat{U}\hat{H}\hat{U}^\dagger+i\hbar\frac{\partial\hat{U}}{\partial t}\hat{U}^\dagger$表象下展开,才能判断是一个什么样的量子态。

        \subsubsection{匀速直线运动情形}

            匀速运动情形下,惯性力为0,于是S'系中的Hamilton量为$\hat{H}'=\hat{U}\hat{H}\hat{U}^\dagger=\frac{(\hat{p}-mv)^2}{2m}+\frac{1}{2}m\omega^2(\hat{x}-vt)^2$。引入瞬时动量算符和坐标算符$\hat{P}=\hat{p}-mv$和$\hat{Q}=\hat{x}-vt$,不难验证对易关系$[\hat{Q},\hat{P}]=i\hbar$,于是$\hat{H}'=\frac{\hat{P}^2}{2m}+\frac{1}{2}m\omega^2\hat{Q}^2$仍然是一个谐振子。因此我们可以引入相应的产生湮灭算符

            \begin{equation}
                \begin{array}{lr}
                    \hat{b}=\frac{1}{\sqrt{2m\hbar\omega}}(m\omega\hat{Q}-i\hat{P})=\frac{1}{\sqrt{2m\hbar\omega}}(m\omega\hat{x}-i\hat{p}-m\omega vt+imv)=\hat{a}+\frac{mv(\omega t-i)}{\sqrt{2m\hbar\omega}} \\
                    \hat{b}^\dagger=\frac{1}{\sqrt{2m\hbar\omega}}(m\omega\hat{Q}+i\hat{P})=\frac{1}{\sqrt{2m\hbar\omega}}(m\omega\hat{x}+i\hat{p}+m\omega vt-imv)=\hat{a}+\frac{mv(\omega t+i)}{\sqrt{2m\hbar\omega}}
                \end{array}
            \end{equation}

            在匀速运动情形下,S'系中的观察者看到的S系的基态变成了$\ket{\alpha}$,其中$\alpha=-vt\sqrt{\frac{m\omega}{2\hbar}}-imv\sqrt{\frac{1}{2m\hbar\omega}}$,即$\hat{a}\ket{\alpha}$。由S'系的湮灭算符与S系的湮灭算符的关系$\hat{b}=\hat{a}+\frac{mv(\omega t-i)}{\sqrt{2m\hbar\omega}}$可知,$\hat{b}\ket{\alpha}=0$,即匀速运动情形下,S系的基态在S'系中仍然是基态,尽管这个基态在$\hat{H}$表象下是一个相干态。


        \subsubsection{匀加速运动情形与Unruh效应}

            然而,在匀加速情形下,由于惯性力的作用,我们将看到S系的基态在S'系中不再是基态了。S'系中的态矢量$\ket{\alpha}$中$\alpha=-\frac{1}{2}at^2\sqrt{\frac{m\omega}{2\hbar}}-imat\sqrt{\frac{1}{2m\hbar\omega}}$,Hamilton量
            
            \begin{equation}
                \begin{split}
                    \hat{H}' &= \hat{U}\hat{H}\hat{U}^\dagger+i\hbar\frac{\partial\hat{U}}{\partial t}\hat{U}^\dagger = \frac{(\hat{p}-mat)^2}{2m}+\frac{1}{2}m\omega^2(\hat{x}-\frac{1}{2}at^2)^2-ma\hat{x} \\
                        &= \frac{(\hat{p}-mat)^2}{2m}+\frac{1}{2}m\omega^2(\hat{x}-\frac{1}{2}at^2-\frac{a}{\omega^2})^2-\frac{1}{2}ma^2t^2-\frac{ma^2}{2\omega^2} \\
                        &= \frac{(\hat{p}-mat)^2}{2m}+\frac{1}{2}m\omega^2(\hat{x}-\frac{1}{2}at^2-\frac{a}{\omega^2})^2+V_0(t)
                \end{split}
            \end{equation}
                
            此时,瞬时动量算符和坐标算符定义为$\hat{P}=\hat{p}-mat$,$\hat{Q}=\hat{x}-\frac{1}{2}at^2-\frac{a}{\omega^2}$,相应的产生湮灭算符为

            \begin{equation}
                \begin{array}{lr}
                    \hat{b}=\hat{a}+\frac{1}{\sqrt{2m\hbar\omega}}(mat^2\omega-imat)+\frac{1}{\sqrt{4m\hbar\omega}}mat \\
                    \hat{b}^\dagger=\hat{a}^\dagger+\frac{1}{\sqrt{2m\hbar\omega}}(mat^2\omega+imat)+\frac{1}{\sqrt{4m\hbar\omega}}mat
                \end{array}
            \end{equation}

            不难验证$(\hat{a}+\frac{1}{\sqrt{2m\hbar\omega}}(mat^2\omega-imat))\ket{\alpha}=0$,即若不考虑惯性力,$\hat{U}\ket{0}$仍然是$\hat{H}'_0=\hat{U}\hat{H}\hat{U}^\dagger$的瞬时基态。然而由于惯性力的作用,$\hat{U}\ket{0}$不再是$\hat{H}'$的瞬时基态。不难验证$\hat{b}(\hat{U}\ket{0})=\frac{mat}{\sqrt{4m\hbar\omega}}(\hat{U}\ket{0})$,即$\hat{U}\ket{0}$是$\hat{H}'$表象下的相干态$\ket{\alpha'}=e^{-\frac{|\alpha'|^2}{2}}\sum_{n=0}^\infty\frac{\alpha'^n}{\sqrt{n!}}\ket{n}$,其中$\alpha'=$。特别地,当$\alpha'=0$时,系统回到了基态。




    \section{总结与讨论}

        本文讨论了量子力学中的经典平动参考系变换,首先说明了参考系变换下量子态的变换可以由一个Weyl平移算子来刻画,其物理意义是经典相空间作为一个加法群在量子态空间上的群表示,而这个群正是Weyl-Heisenberg群相对于U(1)群的商群,同构于$(\mathcal{R}^2,+)$。接下来我们讨论了这个幺正算子所诱导的绝热规范势,指出了它的经典物理意义是惯性力,并由此构建了量子达朗贝尔原理。基于参考系变换下量子态和Hamilton量的变换法则,我们研究了一维谐振子,着重分析了基态的变换,证明了匀速参考系变换下基态仍然变为基态,而匀加速情形下基态将变为相干态。

        这个结论是非平庸的,它意味着在S'参考系中量子态的演化存在激发。这个结论可以运用于量子调控的领域中,我们可以操控变化的电磁场中的一个带电谐振子,它的Hamilton量的形式$\hat{H}=\frac{(\hat{p}+qA(t))^2}{2m}+\frac{1}{2}m\omega^2\hat{x}^2-qE(t)\hat{x}$,并将初态制备为基态。调控合适的电磁场,使Hamilton量的形式与(???)式一致,那么此电磁场下的谐振子将遵循本文的运动规律,即处于相干态$\ket{\alpha}$。特别地,当$\alpha=0$时,系统回到了基态,这正是量子绝热捷径(short-cut to adiabatic)。所以说,作为一个副产品,本文给出了一系列新的量子绝热捷径。

        另一方面,我们知道匀加速观者看到的闵氏时空将表现为Rindler时空,这个时空中的标量场具有Unruh效应。由于标量场的物理图像是无穷多无耦合谐振子,因此根据本文对谐振子的研究,我们也可以在Galileo时空中考虑是否存在Unruh效应。形式上来看,在匀加速观者看来,谐振子处于相干态,因此处在第n能级的概率正比于$(|\alpha|^2)^n$,对比玻尔兹曼因子$e^{-\frac{E_n}{k_BT}}\sim e^{-\frac{n\hbar\omega}{k_BT}}$,我们形式上可以要求

        \begin{equation}
            |\alpha|^2=e^{-\frac{\hbar\omega}{k_BT}}
        \end{equation}

        因此形式上给出一个温度

        \begin{equation}
            k_BT=-\frac{\hbar\omega}{\ln|\alpha|^2}
        \end{equation}

        这显然与Unruh效应所预言的$T\sim a$不同。

        



\end{document}