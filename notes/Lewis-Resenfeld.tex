\documentclass[a4paper]{article}
\usepackage{ctex}
\usepackage{amsmath}
\usepackage{amssymb}

\usepackage{geometry}  
\geometry{left=2.0cm, right=2.0cm, top=2.5cm, bottom=2.5cm}

\usepackage{braket}

\title{Lewis-Resenfeld不变量}
\author{吴梦之}

\begin{document}
    \maketitle

    这篇note旨在梳理Lewis-Resenfeld不变量的主要思想。本文主要参考。

    \section{Lewis-Resenfeld不变量的引入}
        我们的目标是求解一个含时Schrodinger方程

        \begin{equation}
            i\hbar\frac{\partial}{\partial t}\ket{\Psi} = \hat{H}(t)\ket{\Psi}
        \end{equation}

        在这个系统中,我们假定存在一个厄米算符$\hat{I}(t)$,满足

        \begin{equation}
            i\hbar\frac{\partial\hat{I}}{\partial t} + [\hat{I}, \hat{H}] = 0
        \end{equation}

        我们称之是一个不变量。不难验证,$\hat{I}$的一个性质是:若$\ket{\Psi}$是Schrodinger方程的解,那么$\hat{I}\ket{\Psi}$也是Schrodinger方程的解,即

        \begin{equation}
            i\hbar\frac{\partial}{\partial t}(\hat{I}\ket{\Psi}) = \hat{H} (\hat{I}\ket{\Psi})
        \end{equation}

        由于我们假定,每个时刻$\hat{I}(t)$都是一个厄米算符,因此可以认为$\hat{I}(t)$每个时刻的瞬时本征态集合$\{\ket{\lambda(t),\kappa(t)}\}$都构成Hilbert空间的一组完备正交基,其中$\lambda(t)$是$\hat{I}(t)$的瞬时本征值,$\kappa(t)$是简并自由度。方便起见,不妨只考虑无简并的情形,因此$\kappa$自由度就没有必要存在了,因此之后我用$\ket{\phi_n(t)}$来表示$\hat{I}(t)$的瞬时本征态。我们还应指出,$\ket{\phi_n(t)}$并不一定满足Schrodinger方程,它具有一个全局相位的自由度,只有特定相位的态才满足Schrodinger方程,记作

        \begin{equation}
            \ket{\Psi_n(t)}=e^{i\alpha_n(t)}\ket{\phi_n(t)}
        \end{equation}

        在对态做展开时应当用$\ket{\Psi_n(t)}$而不是$\ket{\phi_n(t)}$,这就好比Gauss波包本身并不是含时Schrodinger方程的解,Gauss波包乘上随时间演化的相位以后才是Schrodinger方程的解(尽管这个相位并没有可观测效应)。

        Lewis和Resenfeld证明了两件很重要的事情,第一个是$\hat{I}(t)$的本征值$\lambda(t)$并不随时间变化,即

        \begin{equation}
            \frac{\partial\lambda_n}{\partial t} = 0
        \end{equation}

        第二个重要的事情是相位$\alpha(t)$的演化方程是

        \begin{equation}
            \hbar\frac{d\alpha_n}{dt} = \bra{\phi_n}i\hbar\frac{\partial}{\partial t}-\hat{H}\ket{\phi_n} 
        \end{equation}

        事实上,因子$e^{i\alpha_n(t)}$代表了时域上的一个规范场。最终的解为

        \begin{equation}
            \ket{\Psi(t)} = \Sigma c_n e^{i\alpha(t)} \ket{\phi_n(t)}
        \end{equation}


    \section{Lewis-Resenfeld不变量}

        文献利用Lewis-Resenfeld不变量反解出了$\hat{H}(t)$,思路如下:

        首先利用$\hat{I}(t)$的瞬时本征态的演化写出时间演化算符$\hat{U}(t)$为

        \begin{equation}
            \hat{U}(t) = \sum e^{i\alpha_n(t)}\ket{\phi_n(t)}\bra{\phi_n(0)}
        \end{equation}

        它满足时间演化方程

        \begin{equation}
            i\hbar\frac{\partial\hat{U}}{\partial t} = \hat{H}\hat{U}
        \end{equation}

        于是我们就可以得到$\hat{H}$的表达式为

        \begin{equation}
            \begin{split}
                \hat{H}(t) &= i\hbar\frac{\partial\hat{U}}{\partial t}\hat{U}^\dagger \\
                    &= -\hbar\sum\frac{d\alpha_n}{dt}e^{i\alpha_n(t)}\ket{\phi_n(t)}\bra{\phi_n(t)} + i\hbar\sum\ket{\partial_t\phi_n(t)}\bra{\phi_n(t)}
            \end{split}
        \end{equation}

        在Lie群理论中,$i\hbar\frac{\partial\hat{U}}{\partial t}\hat{U}^\dagger$称作Maurer-Cartan形式。所以从Lie群和微分几何的观点来看,体系的Hamilton量实际上就是U(1)群作用在量子态空间上后的Maurer-Cartan形式的群表示。



    \section{抽象理论观点下的Lewis-Resenfeld不变量}

        量子力学中的物理态空间其实并不是Hilbert空间,而是复射影空间$\mathbb{H}/\mathbb{C}^\times$,其中$\mathbb{H}$是Hilbert空间。特别地,对于有限维Hilbert空间$\mathbb{C}^{n+1}$,其物理态空间是$\mathbb{CP}^n=\mathbb{C}^{n+1}/\mathbb{C}^\times$。例如二能级系统的态空间是$\mathbb{CP}^1$,它微分同胚于单位球面$S^1$,即物理态和$S^1$上的点一一对应,因此我们可以用$S^1$对二能级系统进行可视化描述,这正是Bloch球。
    
        然而,Schrodinger方程和态矢量$\ket{\Psi}$其实也并没有定义在复射影空间$\mathbb{H}/\mathbb{C}^\times$上,而允许一个$U(1)$相位的自由度。因此态矢量其实是定义在$\mathbb{H}/\mathbb{R}^+$,这个$\mathbb{R}^+$代表了概率的归一化。从几何上来看,$\mathbb{H}/\mathbb{R}^+$是$\mathbb{H}/\mathbb{C}^\times$上的纤维丛,可以定义商映射$h:\mathbb{H}/\mathbb{R}^+\to\mathbb{H}/\mathbb{C}^\times$,则这个商映射可以给出Hopf纤维化: $\bigl(\mathbb{H}/\mathbb{R}^+\bigr) / \bigl(\mathbb{H}/\mathbb{C}^\times\bigr) \simeq U(1)$。例如二能级系统的物理态矢量$\ket{\Psi}$定义在$\mathbb{C}^2/\mathbb{R}^+$上,这个空间微分同胚于$S^3$,它相对于态空间$S^2$有一个Hopf纤维化$S^3/S^2\simeq U(1)$。
        
        态的演化实际上是$\mathbb{H}/\mathbb{R}^+$上的一条曲线,其参数是时间t,时间演化算子刻画了这条曲线的演化,即$\ket{\Psi(t)}=\hat{U}(t,t_0)\ket{\Psi(t_0)}$。Hamilton量和时间演化算子的局部关系是$\hat{U}(\delta t) = e^{-\frac{i}{\hbar}\hat{H}(t)\delta t}\sim 1-\frac{i}{\hbar}\hat{H}(t)\delta t$,即它们二者由指数映射相联系。我们知道在微分几何中,指数映射联系着切向量和测地线,因此Hamilton量和时间演化算子具有一定的几何意义。另外,Hamilton量可以写成$\hat{H}=i\hbar\frac{\partial}{\partial t}(log\hat{U})=i\hbar\frac{\partial\hat{U}}{\partial t}\hat{U}^\dagger$,即为Maurer-Cartan形式。

        态矢量$\ket{\Psi(t)}$可以做正交分解,这实际上是指点$\ket{\Psi(t)}$处的切空间存在一个正交标架$\{\ket{e_n}\}$,态矢量可以由这组正交标架分解。也就是说,态矢量既是流形上一个点,又是该点处切空间的一个向量,这与分析力学中的位型空间和广义坐标是相似的。Hamilton量之所以能够表示为一个Maurer-Cartan形式,正是因为它所刻画的是活动标架的运动,这也正是Maurer-Cartan形式的几何意义——刻画曲线上临近两点处的活动标架之间的变换。

        而所谓一个表象Q,实际上是指$\ket{\Psi(t)}$处切空间上的厄米算子$\hat{Q}(t)$,其全体特征向量构成$\ket{\Psi}$处切空间一个正交标架。如果$\hat{Q}(t)$随时间t的变化而变化,那么每一时刻t,$\hat{Q}(t)$都会给出一个正交标架,我们称之为一个活动标架。应当注意的是,t时刻的正交标架,经过$\Delta t$的演化之后得到的正交标架,一般不同于$\hat{Q}(t+\Delta t)$给出的正交标架,接下来我们将看到Lewis-Resenfeld不变量的意义。

        根据陈玺的论文,真正核心的其实是一组基(活动标架)的演化,当我们知道某一组基的演化后,就可以构建时间演化算子,进而构建Hamilton量,其中Hamilton量与时间演化算子的关系由Maurer-Cartan形式来刻画。Lewis-Resenfeld不变量在这个理论中起到脚手架的作用,即Lewis-Resenfeld不变量帮助我们找到了一组基的演化。当然,对于定态问题和绝热过程,我们是已知其一组基的演化的(Hamilton量本征态演化为下一时刻的Hamilton量的本征态),所以这两种情形我们就不需要Lewis-Resenfeld不变量了。当然如果我们非要考虑Lewis-Resenfeld不变量,那么显然Hamilton量本身就是Lewis-Resenfeld不变量。

        我们来进一步考察Lewis-Resenfeld不变量到底干了什么事。若$\ket{\phi_n(t)}$是$\hat{I}(t)$的本征态,即$\hat{I}(t)\ket{\phi_n(t)}=\lambda_n\ket{\phi_n(t)}$,那么经过$\Delta t$的演化,有$\hat{U}(t+\Delta t, t)\ket{\phi_n(t)}\propto\ket{\phi_n(t+\Delta t)}$。但是对于一般的一组基$\ket{\psi_n(t)}$,它们的演化将是$\hat{U}(t+\Delta t, t)\ket{\psi_n(t)}=\sum_m c_{mn}\ket{\psi_m(t+\Delta t)}$。也就是说,某一时刻$\hat{I}(t)$所定义的基(活动标架),经过任意长时间的演化,是那个时刻$\hat{I}(t+\Delta t)$所定义的基(活动标架);而对于一般的表象Q,某一时刻$Q(t)$所定义的基(活动标架),经过一段时间的演化,就不再是那个时刻下$Q(t+\Delta t)$所定义的基(活动标架)了。

\end{document}