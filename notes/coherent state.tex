\documentclass{article}

\usepackage{amsmath}
\usepackage{amssymb}
\usepackage{geometry}
\usepackage{braket}

\geometry{left=2.0cm, right=2.0cm, top=2.5cm, bottom=2.5cm}

\title{Note on coherent state}
\author{Mengzhi Wu}

\begin{document}
    \maketitle

    This note aims to summarize essential properties of the coherent state. I plan to cover these points:

    \begin{itemize}
        \item Basic knowledges on oscillators and coherent states.
        \item Coherent states under Hamiltonian representation and time.
        \item Coherent states under coordinate representation.
        \item Wigner functions for coherent states and time evolution.
        \item classical propertities of coherent states.
    \end{itemize}

    \section{Basic knowledges on oscillators}
        In this section, I'd like to briefly review some basic properties of some operators. The Hamiltonian is $\hat{H}=\frac{\hat{p}^2}{2m}+\frac{1}{2}m\omega^2\hat{x}^2$, and the commutation relation is $[\hat{x},\hat{p}]=i\hbar$(Be careful of the sign!). Then the creation and annihilation operator can be defined as 

        \begin{equation}
            \hat{a}= \frac{1}{\sqrt{2m\hbar\omega}}(i\hat{p}+m\omega\hat{x}), \ \hat{a}^\dagger=\frac{1}{\sqrt{2m\hbar\omega}}(-i\hat{p}+m\omega\hat{x})
        \end{equation}

        And the commutation relation is $[\hat{a}^\dagger, \hat{a}]=-1$(Be careful of the sign!). Inversely, $\hat{x}$, $\hat{p}$ and $\hat{H}$ can be described as 

        \begin{equation}
            \begin{split}
                \hat{x}&= \sqrt{\frac{\hbar}{2m\omega}}(\hat{a}^\dagger+\hat{a}) \\
                \hat{p}&= i\sqrt{\frac{m\hbar\omega}{2}}(\hat{a}^\dagger-\hat{a}) \\
                \hat{H}&= (\hat{a}^+\hat{a}+\frac{1}{2})\hbar\omega
            \end{split}
        \end{equation}

        Suppose the eigenstate of the Hamiltonian is $\ket{n}$ with eigenvalue $E_n=(n+\frac{1}{2})\hbar\omega$ whose wavefunction is 
        
        \begin{equation}
            \Psi_n(x)=\bigl(\frac{m\omega}{\pi\hbar}\bigr)^{\frac{1}{4}}\frac{1}{\sqrt{2^n n!}}H_n(\frac{m\omega}{\hbar}x)\exp\bigl(-\frac{m\omega}{2\hbar}x^2\bigr)
        \end{equation}
        
        Then the creation and annihilation operators give out

        \begin{equation}
            \begin{split}
                \hat{a}\ket{n}&=\sqrt{n}\ket{n-1} \\
                \hat{a}^\dagger\ket{n}&=\sqrt{n+1}\ket{n+1}
            \end{split}
        \end{equation}

        \section{Important opertor formulas}

        Then let's review some useful formulas. The first one is Glauber's formula, based on Baker-Hausdorff formula. This formula claims that if operators $\hat{A}$ and $\hat{B}$ satisfies $[\hat{A}, [\hat{A}, \hat{B}]] = [\hat{B}, [\hat{A}, \hat{B}]] = 0$, then 

        \begin{equation}\label{formula1}
            \exp(\hat{A}+\hat{B}) = \exp(\hat{A})\exp(\hat{B})\exp\biggl(-\frac{1}{2}[\hat{A}, \hat{B}]\biggr)
        \end{equation}

        This formula tell us the commutation property of $\exp(\hat{A})$ and $\exp(\hat{B})$ which is called Weyl commutation relation:
        
        \begin{equation}
            \exp(\hat{A})\exp(\hat{B}) = \exp(\hat{B})\exp(\hat{A})\exp([\hat{A}, \hat{B}])
        \end{equation}

        \vbox{}

        \vbox{}

        The 2nd formula is 

        \begin{equation}\label{formula2}
            \exp(\lambda\hat{A})\hat{B}\exp(-\lambda\hat{A}) = \hat{B} + \lambda [\hat{A},\hat{B}] + \frac{\lambda^2}{2!} [\hat{A}, [\hat{A}, \hat{B}]] + \frac{\lambda^3}{3!}[\hat{A}, [\hat{A}, [\hat{A}, \hat{B}]]] + \dots
        \end{equation}

        Specially, if $[\hat{A}, \hat{B}]=const=:C$, then 

        \begin{equation}
            \exp(\lambda\hat{A})\hat{B}\exp(-\lambda\hat{A}) = \hat{B} + \lambda C
        \end{equation}

        which means a translation of operator $\hat{B}$. More specially, we set $\hat{A}=-\alpha\hat{a}^\dagger+\alpha^*\hat{a}$, $\hat{B}=\hat{a}$ or $\hat{B}=\hat{a}^\dagger$, then we may obtain 

        \begin{equation}\label{translation operator 1}
            \exp(-\alpha\hat{a}^++\alpha^*\hat{a})\hat{a}\exp(\alpha\hat{a}^+-\alpha^*\hat{a}) = \hat{a}+\alpha
        \end{equation}

        \begin{equation}
            \exp(-\alpha\hat{a}^\dagger+\alpha^*\hat{a})\hat{a}^\dagger\exp(\alpha\hat{a}^\dagger-\alpha^*\hat{a}) = \hat{a}^\dagger+\alpha^*
        \end{equation}

        Equation (\ref{translation operator 1}) is very important. Denote $\hat{D}(\alpha)=\exp(\alpha\hat{a}^\dagger -\alpha^*\hat{a})$, then (\ref{translation operator 1}) presents

        \begin{equation}\label{translation operator 2}
            \hat{D}^\dagger(\alpha)\hat{a}\hat{D}(\alpha) = \hat{a} + \alpha
        \end{equation}

        Based on (\ref{translation operator 2}), we may prove that all the eigenstates of $\hat{a}$ are 

        \begin{equation}
            \hat{a} \bigl(\hat{D}(\alpha)\ket{0}\bigr) = \alpha \bigl(\hat{D}(\alpha)\ket{0}\bigr)
        \end{equation}

        Therefore, we may define the coherent states as $\ket{\alpha} = \hat{D}(\alpha)\ket{0}$ with any complex number eigenvalue $\alpha\in\mathbb{C}$, and $\hat{a}\ket{\alpha}=\alpha\ket{\alpha}$. Note that all the translation opertors forms a group called Heisenberg-Weyl group. It's not difficult to find out that Heisenberg-Weyl group is isomorphic to $(\mathbb{C}, +)$, that is 
        
        \begin{equation}        
            \hat{D}(\alpha)\hat{D}(\beta) = \hat{D}(\alpha+\beta)
        \end{equation}

        Since $\hat{D}(\alpha)$ is an operator on a linear space ,the Hilbert space for the oscillator, $\hat{D}(\alpha)$ can be regarded as a group representation of $(\mathbb{C}, +)$.

        \vbox{}

        \vbox{}

        Now that $\hat{a}^\dagger$ and $\hat{a}$ can be described by $\hat{x}$ and $\hat{p}$, we can write the translation operator as another form: 

        \begin{equation}\label{Weyl translation 1}
            \hat{D}(\alpha) = \exp\biggl(i\Im(\alpha)\sqrt{\frac{2m\omega}{\hbar}}\hat{x}-i\Re(\alpha)\sqrt{\frac{2}{m\hbar\omega}}\hat{p}\biggr)
        \end{equation}

        It's not difficult to find that $\hat{D}(\alpha)$ is actually a Weyl translation operator $\hat{W}(\Im(\alpha)\sqrt{\frac{2m\omega}{\hbar}}, \Re(\alpha)\sqrt{\frac{2}{m\hbar\omega}})$. Based on Glauber's formula, we can express $\hat{D}(\alpha)$ as a separated form:

        \begin{equation}\label{Weyl translation 2}
            \hat{D}(\alpha) = \exp\biggl(i\Im(\alpha)\sqrt{\frac{2m\alpha}{\hbar}}\hat{x}\biggr) \exp\biggl(-i\Re(\alpha)\sqrt{\frac{2}{m\hbar\omega}}\hat{p}\biggr) \exp(-i \Im(\alpha)\Re(\alpha))
        \end{equation}

        Recall that all $\hat{D}(\alpha)$ forms a group representation of $(\mathbb{C}, +)$, it can also be considered as a group representation of $(\mathbb{R}^2, +)$, where the $\mathbb{R}^2$ represnts the x and p coordinate, which is actually a point in the phase space. So a coherent state is actually a group action by a point $(\Im(\alpha)\sqrt{\frac{2m\omega}{\hbar}}, \Re(\alpha)\sqrt{\frac{2}{m\hbar\omega}})$ in the phase space, and the evolution of a coherent state is a group action by a trajectory in the phase space. Therefore, we may use points and trajectories to describe coherent states of an oscillator.

        \vbox{}

        \vbox{}

        In the end of this section, I'd like to mention a further appliacation of (\ref{formula2}). Set $\hat{A}=\hat{N}=\hat{a}^\dagger\hat{a}$, and $\hat{B}=\hat{a}^\dagger$ or $\hat{a}$. Then we may obtain

        \begin{equation}
            \exp(\lambda\hat{N})\hat{a}\exp(-\lambda\hat{N}) = \hat{a} e^{-\lambda}
        \end{equation}

        \begin{equation}
            \exp(\lambda\hat{N})\hat{a}^\dagger\exp(-\lambda\hat{N}) = \hat{a}^\dagger e^{\lambda}
        \end{equation}

        Still pay attention to the sign! This formula will give out the squeezed states. 

    


    \section{Hamiltonian representation and time evolution}
    
        It's not difficult to show that the expansion of a coherent state $\ket{\alpha}$ by the eigenstates of the Hamiltonian eigenstates is 

        \begin{equation}
            \ket{\alpha} = e^{-|\alpha|^2/2} \sum \frac{\alpha^n}{\sqrt{n!}}\ket{n}
        \end{equation}

        Therefore the evolution of that state is $\ket{\alpha(t)}=e^{-i\hat{H}t/\hbar}\ket{\alpha}=\ket{\alpha} = e^{-|\alpha|^2/2} \sum \frac{\alpha^n}{\sqrt{n!}}e^{-iE_nt/\hbar}\ket{n}$. Since $E_n=(n+\frac{1}{2})\hbar\omega$, that factor can be written as $e^{-iE_nt/\hbar}=e^{-i\frac{1}{2}\omega t}e^{-in\omega t}=e^{-i\frac{1}{2}\omega t}\bigl(e^{-i\omega t}\bigr)^n$, where the latter phase can be combined with $\alpha^n$ and presents $(\alpha e^{-i\omega t})^n$ and $|\alpha e^{-i\omega t}|^2=|\alpha|^2$. Therefore, 

        \begin{equation}
            \ket{\alpha(t)} = e^{-i\frac{1}{2}\omega t} \exp\bigl(-|\alpha e^{-i\omega t}|^2\bigr) \sum \frac{(\alpha e^{-i\omega})^n}{\sqrt{n!}}\ket{n}
        \end{equation}

        which suggests 

        \begin{equation}
            \ket{\alpha(t)} = e^{-i\frac{1}{2}\omega t} \ket{\alpha e^{-i\omega t}}
        \end{equation}

        Now that the translation operator is a group representation of $(\mathbb{R}^2, +)$, where $\mathbb{R}^2$ is the phase space, the evolution of $\ket{\alpha(t)}$ corresponds to $(\Im(\alpha\exp(-i\omega t))\sqrt{\frac{2m\omega}{\hbar}}, \Re(\alpha\exp(-i\omega t))\sqrt{\frac{2}{m\hbar\omega}})$. So we can use a circle in $\mathbb{C}$ to described the trajectory of $\ket{\alpha(t)}$.

        

    \section{Coherent states under coordinate representation}
        
        Now let's calculate the coherent state under coordinate representation, i.e. wavefunction of the coherent state $\braket{x|\alpha}$. Notice the expression (\ref{Weyl translation 1}) of $\hat{D}(\alpha)$ by $\hat{x}$ and $\hat{p}$, we can consider the operation of $\hat{D}(\alpha)$ over coordinate eigenstate $\ket{x}$. Then  $\braket{x|\alpha}$ can be expressed as $\bra{x}\hat{D}(\alpha)\ket{{0}}$, where the translation operator operates on $\bra{x}$. Note that $\hat{D}(\alpha)^\dagger=\hat{D}(-\alpha)$

        \begin{equation}
            \begin{split}    
                \hat{D}(-\alpha)\ket{x} &= \exp\biggl(i\Im(-\alpha)\sqrt{\frac{2m\omega}{\hbar}}\hat{x}\biggr) \exp\biggl(-i\Re(-\alpha)\sqrt{\frac{2}{m\hbar\omega}}\hat{p}\biggr) \exp(-i \Im(-\alpha)\Re(-\alpha)) \ket{x} \\
                    &= \exp\biggl(-i\Im(\alpha)\sqrt{\frac{2m\omega}{\hbar}}\bigl(x-\Re(\alpha)\sqrt{\frac{2\hbar}{m\omega}}\bigr)\biggr) \exp(-i \Im(\alpha)\Re(\alpha)) \ket{x-\Re(\alpha)\sqrt{\frac{2\hbar}{m\omega}}}  \\
                    &= \exp\biggl(-i\Im(\alpha)\sqrt{\frac{2m\omega}{\hbar}}x\biggr) \exp(i\Im(\alpha)\Re(\alpha)) \ket{x-\Re(\alpha)\sqrt{\frac{2\hbar}{m\omega}}}
            \end{split}
        \end{equation}

        Therefore the wavefunction of a coherent state $\ket{\alpha}$ is 
        
        \begin{equation}
            \begin{split}
                \Psi_\alpha(x) &= \bra{x}\hat{D}(\alpha)\ket{0} = \braket{\hat{D}(-\alpha)x|0} \\
                    &= \exp\biggl(i\Im(\alpha)\sqrt{\frac{2m\omega}{\hbar}}x\biggr) \exp(-i\Im(\alpha)\Re(\alpha)) \braket{x-\Re(\alpha)\sqrt{\frac{2\hbar}{m\omega}} | 0} \\
                    &= N \exp\biggl(i\Im(\alpha)\sqrt{\frac{2m\omega}{\hbar}}x\biggr) \exp(-i\Im(\alpha)\Re(\alpha)) \exp\biggl(-\frac{m\omega}{2\hbar} \bigl(x-\Re(\alpha)\sqrt{\frac{2\hbar}{m\omega}}\bigr)^2 \biggr)
            \end{split}
        \end{equation}

        where N is the normalization constant.

        Since $\ket{\alpha(t)}=e^{-i\frac{1}{2}\omega t}\ket{\alpha e^{-i\omega t}}$, the wavefunction at time t is 

        \begin{equation}
            \begin{split}
                \Psi_\alpha(x,t) &= N \exp\biggl(i\Im(\alpha e^{-i\omega t})\sqrt{\frac{2m\omega}{\hbar}}x\biggr) \exp(-i\Im(\alpha e^{-i\omega t})\Re(\alpha e^{-i\omega t})) \exp\biggl(-\frac{m\omega}{2\hbar} \bigl(x-\Re(\alpha e^{-i\omega t})\sqrt{\frac{2\hbar}{m\omega}}\bigr)^2 \biggr) \exp(-i\frac{1}{2}\omega t) \\
                    &= N \exp\biggl(i\bigl(\Im(\alpha)\cos(\omega t)-\Re(\alpha)\sin(\omega t)\bigr) \sqrt{\frac{2m\omega}{\hbar}}x\biggr) \exp\biggl(-i\bigl(\Re(\alpha)\Im(\alpha)\cos(2\omega t)-\frac{(\Re(\alpha)^2-\Im(\alpha)^2)}{2}\sin(2\omega t)\bigr)\biggr) \\
                    &\times \exp\biggl(-\frac{m\omega}{2\hbar} \bigl(x-\big(\Re(\alpha)\cos(\omega t)+\Im(\alpha)\sin(\omega t)\big)\sqrt{\frac{2\hbar}{m\omega}}\bigr)^2 \biggr) \exp(-i\frac{1}{2}\omega t)
            \end{split}
        \end{equation}
        

    \section{Wigner function for coherent states}

        The Wigner function is defined as 

        \begin{equation}
            W(x, p) = \int_{-\infty}^\infty \frac{d\xi}{2\pi}\bra{x+\frac{1}{2}\xi}\hat{\rho}\ket{x-\frac{1}{2}\xi} e^{-\frac{i}{\hbar}p\xi}
        \end{equation}
        
        where the density matrix $\hat{\rho}=\ket{\alpha}\bra{{\alpha}}$. Therefore, 

        \begin{equation}
            \begin{split}
                W(x, p) &= \int_{-\infty}^\infty\frac{d\xi}{2\pi} \braket{x+\frac{1}{2}\xi|\alpha}\braket{\alpha|x-\frac{1}{2}\xi} e^{-\frac{i}{\hbar}p\xi} \\
                    &= N^2 \int_{-\infty}^\infty\frac{d\xi}{2\pi} e^{i\Im(\alpha)\sqrt{\frac{2m\omega}{\hbar}}\bigl(x+\frac{1}{2}\xi\bigr)} e^{-i\Im(\alpha)\Re(\alpha)} e^{-\frac{m\omega}{2\hbar}\biggl(x+\frac{1}{2}\xi-\Re(\alpha)\sqrt{\frac{2\hbar}{m\omega}}\biggr)^2} \\
                    &\times e^{-i\Im(\alpha)\sqrt{\frac{2m\omega}{\hbar}}\bigl(x-\frac{1}{2}\xi\bigr)} e^{i\Im(\alpha)\Re(\alpha)} e^{-\frac{m\omega}{2\hbar}\biggl(x-\frac{1}{2}\xi-\Re(\alpha)\sqrt{\frac{2\hbar}{m\omega}}\biggr)^2} \times e^{-\frac{i}{\hbar}p\xi} \\
                    &= N^2 \int_{-\infty}^\infty\frac{d\xi}{2\pi} \exp\biggl[i\Im(\alpha)\sqrt{\frac{2m\omega}{\hbar}}  -\frac{m\omega}{2\hbar}\biggl(\bigl(x+\frac{1}{2}\xi-\Re(\alpha)\sqrt{\frac{2\hbar}{m\omega}}\bigr)^2 + \bigl(x-\frac{1}{2}\xi-\Re(\alpha)\sqrt{\frac{2\hbar}{m\omega}}\bigr)^2 \biggr) -\frac{i}{\hbar}p\xi \biggr] \\
                    &= N^2 \int_{-\infty}^\infty\frac{d\xi}{2\pi} \exp\biggl[i\Im(\alpha)\sqrt{\frac{2m\omega}{\hbar}}  -\frac{m\omega}{2\hbar}\biggl(\frac{1}{2}\xi^2 + 2\bigl(x-\Re(\alpha)\sqrt{\frac{2\hbar}{m\omega}}\bigr)^2 \biggr) -\frac{i}{\hbar}p\xi \biggr] \\
                    &= N^2 \exp\biggl(-\frac{m\omega}{\hbar}\bigl(x-\Re(\alpha)\sqrt{\frac{2\hbar}{m\omega}}\bigr)^2\biggr) \int_{-\infty}^\infty\frac{d\xi}{2\pi} \exp\biggl[-\frac{m\omega}{2\hbar}\xi^2+\biggl(\Im(\alpha)\sqrt{\frac{2m\omega}{\hbar}}-\frac{p}{\hbar}\biggr)\xi\biggr] \\
                    &= N^2 \exp\biggl(-\frac{m\omega}{\hbar}\bigl(x-\Re(\alpha)\sqrt{\frac{2\hbar}{m\omega}}\bigr)^2\biggr)  \sqrt{\frac{\hbar}{2\pi m\omega}} \exp\biggl[-\frac{\hbar}{2m\omega}\biggl(\Im(\alpha)\sqrt{\frac{2m\omega}{\hbar}}-\frac{p}{\hbar}\biggr)^2\biggr] \\
                    &= N^2 \sqrt{\frac{\hbar}{2\pi m\omega}} \exp\biggl(-\frac{m\omega}{\hbar}\bigl(x-\Re(\alpha)\sqrt{\frac{2\hbar}{m\omega}}\bigr)^2\biggr) \exp\biggl(-\frac{1}{2m\hbar\omega}\bigl(p-\Im(\alpha)\sqrt{2m\hbar\omega}\bigr)^2\biggr)
            \end{split}
        \end{equation}

        In the 6th "=", we use the formula $\int\frac{dx}{2\pi}e^{-ax^2+ibx} = \frac{e^{-b^2/4a}}{2\sqrt{\pi a}}$. 

        It's not difficult to check the property of the Wigner function that the x and p probability distributions are given by the marginals:

        \begin{equation}
            \int dp W(x, p) = \bra{x}\hat{\rho}\ket{x} \simeq |\Psi(x)|^2
        \end{equation}

        \begin{equation}
            \int dx W(x, p) = \bra{p}\hat{\rho}\ket{p} \simeq |\phi(p)|^2
        \end{equation}


\end{document}