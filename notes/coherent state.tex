\documentclass{article}

\usepackage{amsmath}
\usepackage{amssymb}
\usepackage{geometry}
\usepackage{braket}

\geometry{left=2.0cm, right=2.0cm, top=2.5cm, bottom=2.5cm}

\title{Note on coherent state}
\author{Mengzhi Wu}

\begin{document}
    \maketitle

    This note aims to summarize essential properties of the coherent state. I plan to cover these points:

    \begin{itemize}
        \item Basic propertities for creation and annihilation operators.
        \item Coherent states under Hamiltonian representation.
        \item Coherent states under coordinate representation.
        \item Wigner functions for coherent states and time evolution.
        \item classical propertities of coherent states.
    \end{itemize}

    \section{Basic knowledges on oscillators}
        In this section, I'd like to briefly review some basic properties of some operators. The Hamiltonian is $\hat{H}=\frac{\hat{p}^2}{2m}+\frac{1}{2}m\omega^2\hat{x}^2$, and the commute relations is $[\hat{x},\hat{p}]=i\hbar$(Be careful of the sign!). Then the creation and annihilation operator can be defined as 

        \begin{equation}
            \hat{a}= \frac{1}{\sqrt{2m\hbar\omega}}(i\hat{p}+m\omega\hat{x}), \ \hat{a}^\dagger=\frac{1}{\sqrt{2m\hbar\omega}}(-i\hat{p}+m\omega\hat{x})
        \end{equation}

        And the commute relations is $[\hat{a}^\dagger, \hat{a}]=-1$(Be careful of the sign!). Inversely, $\hat{x}$, $\hat{p}$ and $\hat{H}$ can be described as 

        \begin{equation}
            \begin{split}
                \hat{x}&= \sqrt{\frac{\hbar}{2m\omega}}(\hat{a}^\dagger+\hat{a}) \\
                \hat{p}&= i\sqrt{\frac{m\hbar\omega}{2}}(\hat{a}^\dagger-\hat{a}) \\
                \hat{H}&= (\hat{a}^+\hat{a}+\frac{1}{2})\hbar\omega
            \end{split}
        \end{equation}

        Suppose the eigenstate of the Hamiltonian is $\ket{n}$ with eigenvalue $E_n=(n+\frac{1}{2})\hbar\omega$ whose wavefunction is 
        
        \begin{equation}
            \Psi_n(x)=\bigl(\frac{m\omega}{\pi\hbar}\bigr)^{\frac{1}{4}}\frac{1}{\sqrt{2^n n!}}H_n(\frac{m\omega}{\hbar}x)\exp\bigl(-\frac{m\omega}{2\hbar}x^2\bigr)
        \end{equation}
        
        Then the creation and annihilation operators give out

        \begin{equation}
            \begin{split}
                \hat{a}\ket{n}&=\sqrt{n}\ket{n-1} \\
                \hat{a}^\dagger\ket{n}&=\sqrt{n+1}\ket{n+1}
            \end{split}
        \end{equation}

        \section{Important opertor formulas}

        Then let's review some useful formulas. The first one is Glauber's formula, based on Baker-Hausdorff formula. This formula claims that if operators $\hat{A}$ and $\hat{B}$ satisfies $[\hat{A}, [\hat{A}, \hat{B}]] = [\hat{B}, [\hat{A}, \hat{B}]] = 0$, then 

        \begin{equation}\label{formula1}
            \exp(\hat{A}+\hat{B}) = \exp(\hat{A})\exp(\hat{B})\exp\biggl(-\frac{1}{2}[\hat{A}, \hat{B}]\biggr)
        \end{equation}

        The 2nd formula is 

        \begin{equation}\label{formula2}
            \exp(\lambda\hat{A})\hat{B}\exp(-\lambda\hat{A}) = \hat{B} + \lambda [\hat{A},\hat{B}] + \frac{\lambda^2}{2!} [\hat{A}, [\hat{A}, \hat{B}]] + \frac{\lambda^3}{3!}[\hat{A}, [\hat{A}, [\hat{A}, \hat{B}]]] + \dots
        \end{equation}

        Specially, if $[\hat{A}, \hat{B}]=const=:C$, then 

        \begin{equation}
            \exp(\lambda\hat{A})\hat{B}\exp(-\lambda\hat{A}) = \hat{B} + \lambda C
        \end{equation}

        which means a translation of operator $\hat{B}$. More specially, we set $\hat{A}=-\alpha\hat{a}^\dagger+\alpha^*\hat{a}$, $\hat{B}=\hat{a}$ or $\hat{B}=\hat{a}^\dagger$, then we may obtain 

        \begin{equation}\label{translation operator 1}
            \exp(-\alpha\hat{a}^++\alpha^*\hat{a})\hat{a}\exp(\alpha\hat{a}^+-\alpha^*\hat{a}) = \hat{a}+\alpha
        \end{equation}

        \begin{equation}
            \exp(-\alpha\hat{a}^\dagger+\alpha^*\hat{a})\hat{a}^\dagger\exp(\alpha\hat{a}^\dagger-\alpha^*\hat{a}) = \hat{a}^\dagger+\alpha^*
        \end{equation}

        Equation (\ref{translation operator 1}) is very important. Denote $\hat{D}(\alpha)=\exp(\alpha\hat{a}^\dagger -\alpha^*\hat{a})$, then (\ref{translation operator 1}) presents

        \begin{equation}\label{translation operator 2}
            \hat{D}^\dagger(\alpha)\hat{a}\hat{D}(\alpha) = \hat{a} + \alpha
        \end{equation}

        Based on (\ref{translation operator 2}), we may prove that all the eigenstates of $\hat{a}$ are 

        \begin{equation}
            \hat{a} \bigl(\hat{D}(\alpha)\ket{0}\bigr) = \alpha \bigl(\hat{D}(\alpha)\ket{0}\bigr)
        \end{equation}

        Therefore, we may define the coherent states as $\ket{\alpha} = \hat{D}(\alpha)\ket{0}$ with ANY complex number eigenvalue $\alpha\in\mathbb{C}$, and $\hat{a}\ket{\alpha}=\alpha\ket{\alpha}$.

        Since $\hat{a}^\dagger$ and $\hat{a}$ can be described by $\hat{x}$ and $\hat{p}$, we can write the translation operator as another form: 

        \begin{equation}\label{Weyl translation 1}
            \hat{D}(\alpha) = \exp\biggl(i\Im(\alpha)\sqrt{\frac{2m\omega}{\hbar}}\hat{x}-i\Re(\alpha)\sqrt{\frac{2}{m\hbar\omega}}\hat{p}\biggr)
        \end{equation}

        It's not difficult to find that $\hat{D}(\alpha)$ is actually a Weyl translation operator $\hat{W}(\Im(\alpha)\sqrt{\frac{2m\omega}{\hbar}}, \Re(\alpha)\sqrt{\frac{2}{m\hbar\omega}})$. Based on Glauber's formula, we can express $\hat{D}(\alpha)$ as a separated form:

        \begin{equation}\label{Weyl translation 2}
            \hat{D}(\alpha) = \exp\biggl(i\Im(\alpha)\sqrt{\frac{2m\alpha}{\hbar}}\hat{x}\biggr) \exp\biggl(-i\Re(\alpha)\sqrt{\frac{2}{m\hbar\omega}}\hat{p}\biggr) \exp(-i \Im(\alpha)\Re(\alpha))
        \end{equation}

        \vbox{}

        \vbox{}

        In the end of this section, I'd like to mention a further appliacation of (\ref{formula2}). Set $\hat{A}=\hat{N}=\hat{a}^\dagger\hat{a}$, and $\hat{B}=\hat{a}^\dagger$ or $\hat{a}$. Then we may obtain

        \begin{equation}
            \exp(\lambda\hat{N})\hat{a}\exp(-\lambda\hat{N}) = \hat{a} e^{-\lambda}
        \end{equation}

        \begin{equation}
            \exp(\lambda\hat{N})\hat{a}^\dagger\exp(-\lambda\hat{N}) = \hat{a}^\dagger e^{\lambda}
        \end{equation}

        Still pay attention to the sign! This formula will give out the squeezed states. 




    \section{Coherent states under Hamiltonian and coordinate representation}
        
        Now let's calculate the coherent state under coordinate representation, i.e. wavefunction of the coherent state $\braket{x|\alpha}$. Notice the expression (\ref{Weyl translation 1}) of $\hat{D}(\alpha)$ by $\hat{x}$ and $\hat{p}$, we can consider the operation of $\hat{D}(\alpha)$ over coordinate eigenstate $\ket{x}$. Then  $\braket{x|\alpha}$ can be expressed as $\bra{x}\hat{D}(\alpha)\ket{{0}}$, where the translation operator operates on $\bra{x}$. Note that $\hat{D}(\alpha)^\dagger=\hat{D}(-\alpha)$

        \begin{equation}
            \begin{split}    
                \hat{D}(\alpha)\ket{x} &= \exp\biggl(i\Im(\alpha)\sqrt{\frac{2m\alpha}{\hbar}}\hat{x}\biggr) \exp\biggl(-i\Re(\alpha)\sqrt{\frac{2}{m\hbar\omega}}\hat{p}\biggr) \exp(-i \Im(\alpha)\Re(\alpha)) \ket{x} \\
                    &= \exp\biggl(i\Im(\alpha)\sqrt{\frac{2m\omega}{\hbar}}\bigl(x-\Re(\alpha)\sqrt{\frac{2\hbar}{m\omega}}\bigr)\biggr) \exp(-i \Im(\alpha)\Re(\alpha)) \ket{x-\Re(\alpha)\sqrt{\frac{2\hbar}{m\omega}}} 
            \end{split}
        \end{equation}

        
        
\end{document}